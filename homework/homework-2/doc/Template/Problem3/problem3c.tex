%---------------------------------------------------------------------------------
% QUESTION 3.C
%---------------------------------------------------------------------------------
\secGR{Nonlinear Bicycle Model, Variable Speed}

Finally, let's look at the difference we might get with a nonlinear tire model. Using the same variable speed scheme
from \qrefGR{Linear Bicycle Model, Variable Speed}, set the option flag in \verb!simulate_step! to use the Fiala
tire model instead of the linear model.

Follow the prompts in \GRno{} to create a script which simulates the double lane change maneuver captured in the
provided dataset. You should use the nonlinear bicycle model (Fiala tire model).

Three experiments were run, one at low speed, one at intermediate speed, and one at high speed.  We have separated all
of the data into MATLAB cell arrays, with the first cell corresponding to the low speed test, etc. We have trimmed the
dataset to contain only the maneuver in each cell (trimmed out lining the car back up, etc.) Use the measured longitudinal speed in your equations of motion.

Plot the yaw rate (in deg/s), lateral velocity (in m/s), and lateral acceleration (in \si{\m/\s^2}) from each
experimental run together with the corresponding simulated values to see how well your model predicts the vehicle's
behavior.

\iftoggle{solution}{
    \input{Solutions/matlabgrader.tex}
}

\iftoggle{condensed}{}{
    \newpage
}
