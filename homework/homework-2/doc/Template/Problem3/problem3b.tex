%---------------------------------------------------------------------------------
% QUESTION 3.B
%---------------------------------------------------------------------------------
\secGR{Linear Bicycle Model, Variable Speed}

It is possible to allow the speed to vary in your simulation. We are going to assume that we still have two state
variables (yaw rate and lateral velocity) and simply update the longitudinal velocity with measured data at every time
step.

\textit{NOTE: Here we're fixing the mismatch in timing with the recorded longitudinal velocity in exactly the same way
we handled steer angle data before.}

Follow the prompts in \GRno{} to create a script which simulates the double lane change maneuver captured in the
provided dataset. You should still use the linear bicycle model (linear tire model).

Three experiments were run, one at low speed, one at intermediate speed, and one at high speed.  We have separated all
of the data into MATLAB cell arrays, with the first cell corresponding to the low speed test, etc. We have trimmed the
dataset to contain only the maneuver in each cell (trimmed out lining the car back up, etc.) Use the measured longitudinal speed in your equations of motion in
this simulation instead of the constant speed you assumed previously.

Plot the yaw rate (in deg/s), lateral velocity (in m/s), and lateral acceleration (in \si{\m/\s^2}) from each
experimental run together with the corresponding simulated values to see how well your model predicts the vehicle's
behavior.

\iftoggle{solution}{
    \input{Solutions/matlabgrader.tex}
}

