%---------------------------------------------------------------------------------
% QUESTION 1.D
%--------------------------------------------------------------------------------
\secGR{Taking a Single Time Step in Simulation}

To effectively simulate a vehicle's dynamics, you need to do the following things in order at each time step:

\begin{enumerate}
    \item Calculate slip angles from vehicle states and the steer angle
    \item Calculate tire forces from slip angles
    \item Calculate vehicle state derivatives from equations of motion using current states and tire forces
    \item Update the vehicle states (and steer angle if necessary)
\end{enumerate}

Follow the prompts in \GRno{} to create a function to take one time step in your simulation, updating the vehicle states
$U_y$ and $r$ (we're only looking at velocity states in this assignment). 
Calculate the lateral acceleration $a_y$ in this function as well. We'll use it later in our analysis. We have included
tested versions of the functions you have written so far and the \verb!integrate_euler! function you wrote last week for
you to use here. 

To make this function useful in later parts of the assignment, we want to be able to change tire models simply. To do
this we're including an option flag \verb!tire_mode! as an input to the function. This will check for either
\verb!'linear'! or \verb!'fiala'! and use the respective model. We are only using the linear tire model in this
assignment, so you only need to fill in the section for \verb!'linear'! for now. 

You are to modify the vehicle states and lateral acceleration \verb!Uy_1!, \verb!r_1!, and \verb!a_y! in the function template in \GRno{}.

\iftoggle{solution}{
    \input{Solutions/matlabgrader.tex}
}

\newpage

