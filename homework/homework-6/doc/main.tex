%---------------------------------------------------------------------------------
% HEADER
%---------------------------------------------------------------------------------
% Stanford - ME227 Spring 2021
% Instructor: Chris Gerdes
% TA's: Trey Weber, Lucio Mondavi, Will Harvey, Alaisha Alexander
%
% This is a borrowed LaTeX template file for lecture notes for CS267,
% Applications of Parallel Computing, UCBerkeley EECS Department.
%
% PLEASE MODIFY THIS TEMPLATE ONLY WHERE MARKED WITH "STUDENT:"
%
% The only parts that need to be modified in this document
% are found in each of the invididual prompt files which are found in problem folders (e.g. Problem1/problem1a.tex). 
%


\documentclass[10pt]{exam}

%--------=-------------------------------------------------------------------------
% ADD PACKAGES
%---------------------------------------------------------------------------------
% Built in packages
\usepackage[utf8]{inputenc}
\usepackage{amsmath}
\usepackage[title]{appendix}
\usepackage{empheq}
\usepackage{fmtcount}
\usepackage{graphicx}
\usepackage{hyperref}
\usepackage[numbered,framed]{matlab-prettifier}
\usepackage{minibox}
\usepackage{siunitx}
\usepackage{stackengine}
\usepackage{xcolor}
\usepackage{xpatch}

% set up hyper link colors
\hypersetup{
    colorlinks=true,
    linkcolor=blue,
    filecolor=magenta,      
    urlcolor=cyan,
}

%Disable all warnings issued by latex starting with "You have..."
\usepackage{silence}
\WarningFilter{latex}{You have requested package}

% Custom packages
\usepackage{Utility/undertilde}

% Custom functions and macros

% This is a borrowed LaTeX template file for lecture notes for CS267,
% Applications of Parallel Computing, UCBerkeley EECS Department.
% Now being used for CMU's 10725 Fall 2012 Optimization course
% taught by Geoff Gordon and Ryan Tibshirani.  When preparing 
% LaTeX notes for this class, please use this template.
%
% To familiarize yourself with this template, the body contains
% some examples of its use.  Look them over.  Then you can
% run LaTeX on this file.  After you have LaTeXed this file then
% you can look over the result either by printing it out with
% dvips or using xdvi. "pdflatex template.tex" should also work.

\setlength{\oddsidemargin}{0 in}
\setlength{\evensidemargin}{0 in}
\setlength{\topmargin}{-0.6 in}
\setlength{\textwidth}{6.5 in}
\setlength{\textheight}{8.5 in}
\setlength{\headsep}{0.75 in}
\setlength{\parindent}{0 in}
\setlength{\parskip}{0.1 in}
\extrafootheight{-0.5in}

% flags to display template, solution, or condensed versions
% toggletrue{variable} or togglefalse{variable}
% nothing stopping you from turning all 3 on and getting a 
% super long document
\newtoggle{template}
\newtoggle{solution}
\newtoggle{condensed}
\newtoggle{student}

\newcommand{\template}{
    \toggletrue{template}
    \togglefalse{solution}
    \togglefalse{condensed}
    \togglefalse{student}
}

\newcommand{\solutions}{
    \toggletrue{solution}
    \togglefalse{template}
    \togglefalse{condensed}
    \togglefalse{student}
}

\newcommand{\condensed}{
    \toggletrue{condensed}
    \togglefalse{template}
    \togglefalse{solution}
    \togglefalse{student}
}

\newcommand{\student}{
    \toggletrue{student}
    \togglefalse{template}
    \togglefalse{solution}
    \togglefalse{condensed}
}

%% For hw6
%\newcommand*\circled[1]{\tikz[baseline=(char.base)]{
%    \node[shape=circle,draw,inner sep=2pt] (char) {#1};}}
%
%% The following commands set up the lecnum (lecture number)
%% counter and make various numbering schemes work relative
%% to the lecture number.

\newcounter{lecnum}

%\renewcommand{\thepage}{\thelecnum.\arabic{page}}
%\renewcommand{\thesection}{\thelecnum.\arabic{section}}
\renewcommand{\thesubsection}{\arabic{section}.\Alph{subsection}}
\renewcommand{\theequation}{\thelecnum.\arabic{equation}}
\renewcommand{\thefigure}{\thelecnum.\arabic{figure}}
\renewcommand{\thetable}{\thelecnum.\arabic{table}}

%\newcommand{\cc}{{\circ}}
%\newcommand{\Cf}{{\tilde{C}_{f\cc}}}
%\newcommand{\Cr}{{\tilde{C}_{r\cc}}}
%\newcommand{\Kpf}{{K_{\phi f}}}
%\newcommand{\Kpr}{{K_{\phi r}}}
%\newcommand{\Kp}{{K_{\phi}}}
%\newcommand{\cs}{{c_\mathrm{sky}}}


%
% The following macro is used to generate the header.
%
\newcommand{\lecture}[4]{
   \pagestyle{myheadings}
   \thispagestyle{plain}
   \newpage
   \setcounter{lecnum}{#1}
   \setcounter{page}{1}
   \noindent
   \begin{center}
   \framebox{
      \vbox{\vspace{2mm}
    \hbox to 6.28in { {\bf ME227: Vehicle Dynamics and Control
	\hfill Spring 2021} }
       \vspace{4mm}
       \hbox to 6.28in { {\Large \hfill Lecture #1: #2  \hfill} }
       \vspace{2mm}
       \hbox to 6.28in { {#3 \hfill \textcopyright #4 } }
      \vspace{2mm}}
   }
   \end{center}
   \markboth{Lecture #1: #2}{Lecture #1: #2}

}

\newcommand{\assignment}[4]{
   \pagestyle{myheadings}
   \thispagestyle{plain}
   \newpage
   \setcounter{lecnum}{#1}
   \setcounter{page}{1}
   \noindent
   \begin{center}
   \framebox{
      \vbox{\vspace{2mm}
    \hbox to 6.28in { {\bf ME227: Vehicle Dynamics and Control
	\hfill Spring 2021} }
       \vspace{4mm}
       \hbox to 6.28in { {\Large \hfill Assignment #1: #2  \hfill} }
       \vspace{2mm}
       \hbox to 6.28in { {#3 \hfill \textcopyright #4 } }
      \vspace{2mm}}
   }
   \end{center}
   \markboth{Assignment #1: #2}{Assignment #1: #2}
}

\newcommand{\solutionME}[4]{
   \pagestyle{myheadings}
   \thispagestyle{plain}
   \newpage
   \setcounter{lecnum}{#1}
   \setcounter{page}{1}
   \noindent
   \begin{center}
   \framebox{
      \vbox{\vspace{2mm}
    \hbox to 6.28in { {\bf ME227: Vehicle Dynamics and Control
	\hfill Spring 2017} }
       \vspace{4mm}
       \hbox to 6.28in { {\Large \hfill Solution #1: #2  \hfill} }
       \vspace{2mm}
       \hbox to 6.28in { {#3 \hfill \textcopyright #4 } }
      \vspace{2mm}}
   }
   \end{center}
   \markboth{Solution #1: #2}{Solution #1: #2}
}

\newcommand{\project}[4]{
   \pagestyle{myheadings}
   \thispagestyle{plain}
   \newpage
   \setcounter{lecnum}{#1}
   \setcounter{page}{1}
   \noindent
   \begin{center}
   \framebox{
      \vbox{\vspace{2mm}
    \hbox to 6.28in { {\bf ME227: Vehicle Dynamics and Control
	\hfill Spring 2020} }
       \vspace{4mm}
       \hbox to 6.28in { {\Large \hfill Project: #2  \hfill} }
       \vspace{2mm}
       \hbox to 6.28in { {#3 \hfill \textcopyright #4 } }
      \vspace{2mm}}
   }
   \end{center}
   \markboth{Project: #2}{Project: #2}
}

%
% Convention for citations is authors' initials followed by the year.
% For example, to cite a paper by Leighton and Maggs you would type
% \cite{LM89}, and to cite a paper by Strassen you would type \cite{S69}.
% (To avoid bibliography problems, for now we redefine the \cite command.)
% Also commands that create a suitable format for the reference list.
\renewcommand{\cite}[1]{[#1]}
\def\beginrefs{\begin{list}%
        {[\arabic{equation}]}{\usecounter{equation}
         \setlength{\leftmargin}{2.0truecm}\setlength{\labelsep}{0.4truecm}%
         \setlength{\labelwidth}{1.6truecm}}}
\def\endrefs{\end{list}}
\def\bibentry#1{\item[\hbox{[#1]}]}

%Use this command for a figure; it puts a figure in wherever you want it.
%usage: \fig{NUMBER}{SPACE-IN-INCHES}{CAPTION}

% Use these for theorems, lemmas, proofs, etc.
\newtheorem{theorem}{Theorem}[lecnum]
\newtheorem{lemma}[theorem]{Lemma}
\newtheorem{proposition}[theorem]{Proposition}
\newtheorem{claim}[theorem]{Claim}
\newtheorem{corollary}[theorem]{Corollary}
\newtheorem{definition}[theorem]{Definition}
\newenvironment{proof}{{\bf Proof:}}{\hfill\rule{2mm}{2mm}}

% **** IF YOU WANT TO DEFINE ADDITIONAL MACROS FOR YOURSELF, PUT THEM HERE:

\newcommand\E{\mathbb{E}}

\newcommand{\eX}{{\boldsymbol{\hat{e}_X}}}
\newcommand{\eY}{{\boldsymbol{\hat{e}_Y}}}
\newcommand{\eZ}{{\boldsymbol{\hat{e}_Z}}}
\newcommand{\pVE}{{\psi_{VE}}}
\newcommand{\pVEdot}{\dot{\psi}_{VE}}

%\newcommand{\roa}{\uvec{r}_{oa}}

% basis vector: arg1 is name
\newcommand{\basvec}[1]{\boldsymbol{\hat{e}_{#1}}}

% "radius" vector (position vector named r) from arg1 to arg2

\renewcommand{\vec}[1]{%
  \smash{\ensurestackMath{\stackengine{1pt}{#1}{\scriptscriptstyle\sim}{U}{c}{F}{F}{S}}}
  \vphantom{#1}
}
\newcommand{\roa}{\vec{r}_{oa}}
\newcommand{\rv}[3]{\vec{r}_{#1#2,#3}}


% similar with velocity and acceleration vectors
\newcommand{\vv}[3]{\vec{v}_{#1#2,#3}}
\newcommand{\wv}[3]{\vec{\omega}_{#1#2,#3}}
\newcommand{\av}[3]{\vec{a}_{#1#2,#3}}

% vector derivatives
\newcommand{\drv}[3]{\vec{\dot{r}}_{#1#2,#3}}
\newcommand{\dvv}[3]{\vec{\dot{v}}_{#1#2,#3}}
\newcommand{\dwv}[3]{\vec{\dot{\omega}}_{#1#2,#3}}

% forces
\newcommand{\F}[2]{\text{F}_\text{#1#2}}

% cornering stiffnesses and slip angles
\newcommand{\Caf}{C_{\alpha f}}
\newcommand{\Car}{C_{\alpha r}}
\newcommand{\af}{\alpha f}
\newcommand{\ar}{\alpha r}

% vehicle states and derivatives
\newcommand{\Uxdot}{\dot{U}_x}
\newcommand{\Uydot}{\dot{U}_y}
\newcommand{\rdot}{\dot{r}}
\newcommand{\dpsi}{\Delta\Psi}
\newcommand{\dpsidot}{\Delta\dot{\Psi}}
\newcommand{\edot}{\dot{e}}

% Gains
\newcommand{\Kla}{K_{la}}
\newcommand{\xla}{x_{la}}

% Commands for Gradescope and Grader
\newcommand{\GS}{\textbf{Gradescope}}
\newcommand{\GR}{\textbf{MATLAB Grader}}
\newcommand{\MO}{\textbf{MATLAB Online}}
\newcommand{\GSno}{Gradescope}
\newcommand{\GRno}{MATLAB Grader}
\newcommand{\MOno}{MATLAB Online}
\newcommand{\M}{MATLAB}
% Commands for numberless section headings while incrementing counter
\newcommand{\secNoNum}[1]{\refstepcounter{section}\section*{#1}}
\newcommand{\subsecNoNum}[1]{\refstepcounter{subsection}\subsection*{#1}}
\newcommand{\ph}{Problem \thesection}
\newcommand{\qh}{Question \thesection.\Alph{subsection}}
\newcommand{\apph}{Appendix \Alph{section}}

% Commands for questions subsection headings
\newcommand{\secPr}[1]{%
    \secNoNum{%
        \ph{} -- #1
    }
    \label{sec:#1}
}

\newcommand{\secApp}[1]{%
    \secNoNum{%
        \apph{} -- #1
    }
    \label{appendix:#1}
}

\newcommand{\secGR}[1]{%
    \subsecNoNum{%
        \color{blue}
        \qh{} -- #1 (\GRno{})
    }
    \label{subsecGR:#1}
}

\newcommand{\secMOGR}[1]{%
    \subsecNoNum{%
        \color{blue}
        \qh{} -- #1 (\MOno{} / \GRno{})
    }
    \label{subsecMOGR:#1}
}

\newcommand{\secMOGS}[1]{%
    \subsecNoNum{%
        \qh{} -- #1 (\MOno{} / \GSno{})
    }
    \label{subsecMOGS:#1}
}

\newcommand{\secGS}[1]{%
    \subsecNoNum{%
        \qh{} -- #1 (\GSno{})
    }
    \label{subsecGS:#1}
}

% Reference other question subsections
\newcommand{\qrefGR}[1]{%
    \textbf{%
        Question \ref{subsecGR:#1}%
    }%
}

\newcommand{\qrefGS}[1]{%
    \textbf{%
        Question \ref{subsecGS:#1}%
    }%
}

\newcommand{\qrefMOGR}[1]{%
    \textbf{%
        Question \ref{subsecMOGR:#1}%
    }%
}

\newcommand{\qrefMOGS}[1]{%
    \textbf{%
        Question \ref{subsecMOGS:#1}%
    }%
}

\newcommand{\pref}[1]{%
    \textbf{%
        Problem \ref{sec:#1}%
    }%
}


% Shortcut for equation
\newcommand{\eq}[1]{%
    \begin{equation*}
        #1
    \end{equation*}
}

% Shortcut for align
\newcommand{\al}[1]{%
    \begin{align*}
        #1
    \end{align*}
}

% Shortcut for problem expectation
\newcommand{\expect}[1]{%
    \textbf{%
        \textit{%
            #1%
        }%
    }%
}

% Shortcut for bold
\newcommand{\bld}[1]{\textbf{#1}}

% Change solution title
\renewcommand{\solutiontitle}{\noindent\enspace}

% Matlab code helper macros
\newcommand{\matin}[1]{\lstinline[style=Matlab-editor]{#1}}
\newcommand{\matld}[1]{\lstinputlisting[style=Matlab-editor, basicstyle=\footnotesize]{#1}}

% Remove question numbers
\qformat{}
\renewcommand{\questionshook}{%
\setlength{\leftmargin}{0pt}%
\setlength{\labelwidth}{-\labelsep}%
}

% Fix duplicate question labeling
\makeatletter
\xpatchcmd{\questions}
  {question@\arabic{question}}
  {question@\arabic{page}@\arabic{question}}
  {}{}
\makeatother

% Shortcut for underline and bold
\newcommand{\ubf}[1]{\underline{\textbf{#1}}}



% Select mode (select only one at a time)
%\template
\condensed
%\solutions
%\student

%---------------------------------------------------------------------------------
% BEGIN DOCUMENT & INTRO
%---------------------------------------------------------------------------------
\begin{document}
\assignment{6}{Linearization and Project Reflection}{Due Thursday, May 27 at 5:00pm PST}{Chris Gerdes, 2017}

\hspace{0.5cm}

\section*{Purpose}

For this homework, we will study linearization, reflect on the project, and start thinking about racing.

\section*{Instructions}

This homework assignment will be submitted using \GSno{}.

All written portions must be turned in through Gradescope. See the Piazza post on homework guidelines for more
instructions on the different homework resources available to you. Whatever format you decide to use, please \fbox{\textbf{BOX}} all of your final answers.



\newpage


%---------------------------------------------------------------------------------
% PROBLEM 1
%---------------------------------------------------------------------------------
% Problem Intro
\secPr{Weight Distribution and Cornering Stiffness}

Both tire corning stiffness and weight distribution play a role in determining the car’s understeer gradient.  While some properties depend only upon the understeer gradient, other properties of the response depend upon the specific combination of weight balance and cornering stiffnesses that produced that understeer gradient.  Let’s look at how cars with the same understeer gradient can vary in transient performance. 

\iftoggle{condensed}{}{
    \vspace*{1cm}
}


% Student Prompts and Responses
%---------------------------------------------------------------------------------
% QUESTION 1.A
%---------------------------------------------------------------------------------
\secGR{Determining an Appropriate Lookahead Gain}

Let's start with the "lookdown" controller ($x_{la} = 0$). Our first step is to determine a gain that seems physically
reasonable.

Follow the prompts in \GRno{} to create a script to calculate $K_{la}$ such that our controller produces a steer 
angle of 3 degrees when the lateral error is one meter and the vehicle is pointed straight along the path.

You are to modify the lookahead gain, \verb!K_la!, to the appropriate value.

\vspace*{0.5cm}

\iftoggle{solution}{
    \input{Solutions/matlabgrader.tex}
}


%---------------------------------------------------------------------------------
% QUESTION 1.B
%---------------------------------------------------------------------------------

\secGR{Incorporating Path Tracking}

Drifting by itself is pretty cool, but what if we wanted to incorporate a reference path? Let's define a new state $x = [r, U_y, e, \Delta\psi]^T$.

%Since we have seen a lot of success with the lookahead controller, let's try adding an additional term to our steering command from Problem 2E:
%$$\delta=k_r(r_{\text{des}}-r)+k_y(U_{y,\text{des}}-U_y)+\delta_{ff}+\delta_{lookahead}$$
%where $\delta_{lookahead}$ is the lookahead controller we have been using all quarter (do not include curvature compensation).

\textbf{Find a value for $\Delta \psi_{eq}$ given the drift equilibrium we are using in this problem. Do not make any small angle assumptions.}

Let's linearize these dynamics as before and then try to find gains that stabilize the system. This time, our linearized matrices are:

\begin{equation}
A = 
\begin{bmatrix}
\frac{\partial \dot r}{\partial r}  & \frac{\partial \dot r}{\partial U_y}  & \frac{\partial \dot r}{\partial e} & \frac{\partial \dot r}{\partial \Delta\psi} \\
\frac{\partial \dot U_y}{\partial r}  & \frac{\partial \dot U_y}{\partial U_y}  & \frac{\partial \dot U_y}{\partial e} & \frac{\partial \dot U_y}{\partial \Delta\psi} \\
\frac{\partial \dot e}{\partial r}  & \frac{\partial \dot e}{\partial U_y}  & \frac{\partial \dot e}{\partial e} & \frac{\partial \dot e}{\partial \Delta\psi} \\
\frac{\partial \dot{\Delta\psi}}{\partial r}  & \frac{\partial \dot{\Delta\psi}}{\partial U_y}  & \frac{\partial \dot{\Delta\psi}}{\partial e} & \frac{\partial \dot{\Delta\psi}}{\partial \Delta\psi} \\
\end{bmatrix}_{x=x_{eq}} 
\end{equation}

\begin{equation}
B = \begin{bmatrix}
\frac{\partial \dot r}{\partial \delta} \\
\frac{\partial \dot U_y}{\partial \delta}\\
\frac{\partial \dot e}{\partial \delta} \\
\frac{\partial \dot \Delta\psi}{\partial \delta}\\
\end{bmatrix}_{x=x_{eq}} 
\end{equation}


\textbf{Find $\Delta{\psi_{ss}}$ and the new linearized dynamics matrices. Using full state feedback as before, place the poles at [-4 +/- 4.5j, -0.15 +/- 0.75j]. You are encouraged to use the 'place' function in MATLAB. Report your gain vector and $\Delta{\psi_{ss}}$ on MATLAB Grader}





\iftoggle{condensed}{
    \vspace*{0.5cm}
}{
    \subsubsection*{Solution:}
}

\iftoggle{solution}{
    %---------------------------------------------------------------------------------
% QUESTION 1.B
%---------------------------------------------------------------------------------

\secGR{Incorporating Path Tracking}

Drifting by itself is pretty cool, but what if we wanted to incorporate a reference path? Let's define a new state $x = [r, U_y, e, \Delta\psi]^T$.

%Since we have seen a lot of success with the lookahead controller, let's try adding an additional term to our steering command from Problem 2E:
%$$\delta=k_r(r_{\text{des}}-r)+k_y(U_{y,\text{des}}-U_y)+\delta_{ff}+\delta_{lookahead}$$
%where $\delta_{lookahead}$ is the lookahead controller we have been using all quarter (do not include curvature compensation).

\textbf{Find a value for $\Delta \psi_{eq}$ given the drift equilibrium we are using in this problem. Do not make any small angle assumptions.}

Let's linearize these dynamics as before and then try to find gains that stabilize the system. This time, our linearized matrices are:

\begin{equation}
A = 
\begin{bmatrix}
\frac{\partial \dot r}{\partial r}  & \frac{\partial \dot r}{\partial U_y}  & \frac{\partial \dot r}{\partial e} & \frac{\partial \dot r}{\partial \Delta\psi} \\
\frac{\partial \dot U_y}{\partial r}  & \frac{\partial \dot U_y}{\partial U_y}  & \frac{\partial \dot U_y}{\partial e} & \frac{\partial \dot U_y}{\partial \Delta\psi} \\
\frac{\partial \dot e}{\partial r}  & \frac{\partial \dot e}{\partial U_y}  & \frac{\partial \dot e}{\partial e} & \frac{\partial \dot e}{\partial \Delta\psi} \\
\frac{\partial \dot{\Delta\psi}}{\partial r}  & \frac{\partial \dot{\Delta\psi}}{\partial U_y}  & \frac{\partial \dot{\Delta\psi}}{\partial e} & \frac{\partial \dot{\Delta\psi}}{\partial \Delta\psi} \\
\end{bmatrix}_{x=x_{eq}} 
\end{equation}

\begin{equation}
B = \begin{bmatrix}
\frac{\partial \dot r}{\partial \delta} \\
\frac{\partial \dot U_y}{\partial \delta}\\
\frac{\partial \dot e}{\partial \delta} \\
\frac{\partial \dot \Delta\psi}{\partial \delta}\\
\end{bmatrix}_{x=x_{eq}} 
\end{equation}


\textbf{Find $\Delta{\psi_{ss}}$ and the new linearized dynamics matrices. Using full state feedback as before, place the poles at [-4 +/- 4.5j, -0.15 +/- 0.75j]. You are encouraged to use the 'place' function in MATLAB. Report your gain vector and $\Delta{\psi_{ss}}$ on MATLAB Grader}





\iftoggle{condensed}{
    \vspace*{0.5cm}
}{
    \subsubsection*{Solution:}
}

\iftoggle{solution}{
    %---------------------------------------------------------------------------------
% QUESTION 1.B
%---------------------------------------------------------------------------------

\secGR{Incorporating Path Tracking}

Drifting by itself is pretty cool, but what if we wanted to incorporate a reference path? Let's define a new state $x = [r, U_y, e, \Delta\psi]^T$.

%Since we have seen a lot of success with the lookahead controller, let's try adding an additional term to our steering command from Problem 2E:
%$$\delta=k_r(r_{\text{des}}-r)+k_y(U_{y,\text{des}}-U_y)+\delta_{ff}+\delta_{lookahead}$$
%where $\delta_{lookahead}$ is the lookahead controller we have been using all quarter (do not include curvature compensation).

\textbf{Find a value for $\Delta \psi_{eq}$ given the drift equilibrium we are using in this problem. Do not make any small angle assumptions.}

Let's linearize these dynamics as before and then try to find gains that stabilize the system. This time, our linearized matrices are:

\begin{equation}
A = 
\begin{bmatrix}
\frac{\partial \dot r}{\partial r}  & \frac{\partial \dot r}{\partial U_y}  & \frac{\partial \dot r}{\partial e} & \frac{\partial \dot r}{\partial \Delta\psi} \\
\frac{\partial \dot U_y}{\partial r}  & \frac{\partial \dot U_y}{\partial U_y}  & \frac{\partial \dot U_y}{\partial e} & \frac{\partial \dot U_y}{\partial \Delta\psi} \\
\frac{\partial \dot e}{\partial r}  & \frac{\partial \dot e}{\partial U_y}  & \frac{\partial \dot e}{\partial e} & \frac{\partial \dot e}{\partial \Delta\psi} \\
\frac{\partial \dot{\Delta\psi}}{\partial r}  & \frac{\partial \dot{\Delta\psi}}{\partial U_y}  & \frac{\partial \dot{\Delta\psi}}{\partial e} & \frac{\partial \dot{\Delta\psi}}{\partial \Delta\psi} \\
\end{bmatrix}_{x=x_{eq}} 
\end{equation}

\begin{equation}
B = \begin{bmatrix}
\frac{\partial \dot r}{\partial \delta} \\
\frac{\partial \dot U_y}{\partial \delta}\\
\frac{\partial \dot e}{\partial \delta} \\
\frac{\partial \dot \Delta\psi}{\partial \delta}\\
\end{bmatrix}_{x=x_{eq}} 
\end{equation}


\textbf{Find $\Delta{\psi_{ss}}$ and the new linearized dynamics matrices. Using full state feedback as before, place the poles at [-4 +/- 4.5j, -0.15 +/- 0.75j]. You are encouraged to use the 'place' function in MATLAB. Report your gain vector and $\Delta{\psi_{ss}}$ on MATLAB Grader}





\iftoggle{condensed}{
    \vspace*{0.5cm}
}{
    \subsubsection*{Solution:}
}

\iftoggle{solution}{
    \input{Solutions/Problem1/problem1b.tex}
}

\iftoggle{template}{
    \begin{solutionorbox}[5in]
    \end{solutionorbox}
    
    \newpage
}

\iftoggle{student}{
%---------------------------------------------------------------------------------
% STUDENT: BEGIN WORK
%---------------------------------------------------------------------------------
% Please box final answer

%---------------------------------------------------------------------------------
% STUDENT: END WORK
%---------------------------------------------------------------------------------
\newpage
}


}

\iftoggle{template}{
    \begin{solutionorbox}[5in]
    \end{solutionorbox}
    
    \newpage
}

\iftoggle{student}{
%---------------------------------------------------------------------------------
% STUDENT: BEGIN WORK
%---------------------------------------------------------------------------------
% Please box final answer

%---------------------------------------------------------------------------------
% STUDENT: END WORK
%---------------------------------------------------------------------------------
\newpage
}


}

\iftoggle{template}{
    \begin{solutionorbox}[5in]
    \end{solutionorbox}
    
    \newpage
}

\iftoggle{student}{
%---------------------------------------------------------------------------------
% STUDENT: BEGIN WORK
%---------------------------------------------------------------------------------
% Please box final answer

%---------------------------------------------------------------------------------
% STUDENT: END WORK
%---------------------------------------------------------------------------------
\newpage
}


%---------------------------------------------------------------------------------
% QUESTION 1.C
%--------------------------------------------------------------------------------
\secGS{Nonlinear Tire Phase Portrait}

Generate a phase portrait for $U_y$ and $r$ and include your plot.  How many equilibria are there? How would you classify these equilibria? Are they stable? Explain why.

\vspace*{0.5cm}

\expect{Include your phase potrait. How many equilibria are there? Describe what types of equilibria these are and if they are stable. Explain your reasoning.}


\iftoggle{condensed}{
    \vspace*{0.5cm}
}{
    \subsubsection*{Solution:}
}

\iftoggle{solution}{
    %---------------------------------------------------------------------------------
% QUESTION 1.C
%--------------------------------------------------------------------------------
\secGS{Nonlinear Tire Phase Portrait}

Generate a phase portrait for $U_y$ and $r$ and include your plot.  How many equilibria are there? How would you classify these equilibria? Are they stable? Explain why.

\vspace*{0.5cm}

\expect{Include your phase potrait. How many equilibria are there? Describe what types of equilibria these are and if they are stable. Explain your reasoning.}


\iftoggle{condensed}{
    \vspace*{0.5cm}
}{
    \subsubsection*{Solution:}
}

\iftoggle{solution}{
    %---------------------------------------------------------------------------------
% QUESTION 1.C
%--------------------------------------------------------------------------------
\secGS{Nonlinear Tire Phase Portrait}

Generate a phase portrait for $U_y$ and $r$ and include your plot.  How many equilibria are there? How would you classify these equilibria? Are they stable? Explain why.

\vspace*{0.5cm}

\expect{Include your phase potrait. How many equilibria are there? Describe what types of equilibria these are and if they are stable. Explain your reasoning.}


\iftoggle{condensed}{
    \vspace*{0.5cm}
}{
    \subsubsection*{Solution:}
}

\iftoggle{solution}{
    \input{Solutions/Problem1/problem1c.tex}
}

\iftoggle{template}{
    \begin{solutionorbox}[3.5in]
    \end{solutionorbox}
}

\iftoggle{student}{
%---------------------------------------------------------------------------------
% STUDENT: BEGIN WORK
%---------------------------------------------------------------------------------
% Please box final answer

%---------------------------------------------------------------------------------
% STUDENT: END WORK
%---------------------------------------------------------------------------------
}

\newpage


}

\iftoggle{template}{
    \begin{solutionorbox}[3.5in]
    \end{solutionorbox}
}

\iftoggle{student}{
%---------------------------------------------------------------------------------
% STUDENT: BEGIN WORK
%---------------------------------------------------------------------------------
% Please box final answer

%---------------------------------------------------------------------------------
% STUDENT: END WORK
%---------------------------------------------------------------------------------
}

\newpage


}

\iftoggle{template}{
    \begin{solutionorbox}[3.5in]
    \end{solutionorbox}
}

\iftoggle{student}{
%---------------------------------------------------------------------------------
% STUDENT: BEGIN WORK
%---------------------------------------------------------------------------------
% Please box final answer

%---------------------------------------------------------------------------------
% STUDENT: END WORK
%---------------------------------------------------------------------------------
}

\newpage



\newpage

%---------------------------------------------------------------------------------
% PROBLEM 2
%---------------------------------------------------------------------------------
% Problem Intro
\secPr{Simulating Vehicles with Different Responses}

In this problem, you will implement a simulation to compare the two cars we have looked at in response to a step steering input. Step inputs are common ways to characterize dynamic systems (both open and closed loop).

\iftoggle{condensed}{}{
    \vspace*{0.5cm}
}

% Student Prompts and Responses
%---------------------------------------------------------------------------------
% PROBLEM 2 - PART A
%---------------------------------------------------------------------------------
\secGR{Simulation}
One way to think about the handling characteristics of a car is that a better handling car more closely tracks the driver’s steering input.  Follow the prompt in \GRno{} to simulate the response of both of these vehicles to a steep steer of 1 degree at 65mph using the simulation you developed last week with the linear tire model and small angle approximations for 4 seconds and plot the yaw rate. 

\iftoggle{condensed}{}{
    \vspace*{0.5cm}
}

\iftoggle{solution}{
    %---------------------------------------------------------------------------------
% PROBLEM 2 - PART A
%---------------------------------------------------------------------------------
\secGR{Simulation}
One way to think about the handling characteristics of a car is that a better handling car more closely tracks the driver’s steering input.  Follow the prompt in \GRno{} to simulate the response of both of these vehicles to a steep steer of 1 degree at 65mph using the simulation you developed last week with the linear tire model and small angle approximations for 4 seconds and plot the yaw rate. 

\iftoggle{condensed}{}{
    \vspace*{0.5cm}
}

\iftoggle{solution}{
    %---------------------------------------------------------------------------------
% PROBLEM 2 - PART A
%---------------------------------------------------------------------------------
\secGR{Simulation}
One way to think about the handling characteristics of a car is that a better handling car more closely tracks the driver’s steering input.  Follow the prompt in \GRno{} to simulate the response of both of these vehicles to a steep steer of 1 degree at 65mph using the simulation you developed last week with the linear tire model and small angle approximations for 4 seconds and plot the yaw rate. 

\iftoggle{condensed}{}{
    \vspace*{0.5cm}
}

\iftoggle{solution}{
    \input{Solutions/Problem2/problem2a.tex}
}

\iftoggle{condensed}{}{
    \newpage
}

}

\iftoggle{condensed}{}{
    \newpage
}

}

\iftoggle{condensed}{}{
    \newpage
}

%---------------------------------------------------------------------------------
% QUESTION 2.B
%---------------------------------------------------------------------------------
\secGS{Reflection of Group Controller}

Now let's move on to your group's controller. Let's take a look at your code once more to make sure that everything was actually implemented correctly. 
Let's again consider whether you think the amount of feedback and feedforward was appropriate in this case.

\textbf{(1) Was everything implemented correctly? If not, what implementation challenges did your team run into?}

\textbf{(2) Did each of your controllers effectively use feedback and feedforward? Support your conclusion with plots of your performance.}


\iftoggle{condensed}{
    \vspace*{0.5cm}
}{
    \subsubsection*{Solution:}
}

\iftoggle{solution}{
    %%---------------------------------------------------------------------------------
% QUESTION 1.A
%---------------------------------------------------------------------------------
\secGR{Determining an Appropriate Lookahead Gain}

Let's start with the "lookdown" controller ($x_{la} = 0$). Our first step is to determine a gain that seems physically
reasonable.

Follow the prompts in \GRno{} to create a script to calculate $K_{la}$ such that our controller produces a steer 
angle of 3 degrees when the lateral error is one meter and the vehicle is pointed straight along the path.

You are to modify the lookahead gain, \verb!K_la!, to the appropriate value.

\vspace*{0.5cm}

\iftoggle{solution}{
    \input{Solutions/matlabgrader.tex}
}


}

\iftoggle{template}{
    \begin{solutionorbox}[5in]
    \end{solutionorbox}
    
    \newpage
}

\iftoggle{student}{
%---------------------------------------------------------------------------------
% STUDENT: BEGIN WORK
%---------------------------------------------------------------------------------
% Please box final answer

%---------------------------------------------------------------------------------
% STUDENT: END WORK
%---------------------------------------------------------------------------------
\newpage
}
%---------------------------------------------------------------------------------
% QUESTION 2.C
%---------------------------------------------------------------------------------
\secGS{Simulating from Equilibrium}

For this problem we will use the simple coupled tire model where $F_x$ is assumed to be known. Incorporate this into your simulation (We will supply verification code). Run your simulation of the three-state bicycle model for 4 seconds using $\delta$ = -10° and $F_{xr}$ equal to the value you computed in Problem 2B. Use the drift equilibrium found in Problem 1D as the initial condition. Plot $U_y$, $r$, and $U_x$ on the same plot. Does Marty hold the drift? What happens? Did you expect this? If you'd like, visualize using \texttt{animateDrift.m}.

\vspace*{0.5cm}


\expect{Include your plot and an explanation of what you see.}


\iftoggle{condensed}{
    \vspace*{0.5cm}
}{
    \subsubsection*{Solution:}
}

\iftoggle{solution}{
    %---------------------------------------------------------------------------------
% QUESTION 2.C
%---------------------------------------------------------------------------------
\secGS{Simulating from Equilibrium}

For this problem we will use the simple coupled tire model where $F_x$ is assumed to be known. Incorporate this into your simulation (We will supply verification code). Run your simulation of the three-state bicycle model for 4 seconds using $\delta$ = -10° and $F_{xr}$ equal to the value you computed in Problem 2B. Use the drift equilibrium found in Problem 1D as the initial condition. Plot $U_y$, $r$, and $U_x$ on the same plot. Does Marty hold the drift? What happens? Did you expect this? If you'd like, visualize using \texttt{animateDrift.m}.

\vspace*{0.5cm}


\expect{Include your plot and an explanation of what you see.}


\iftoggle{condensed}{
    \vspace*{0.5cm}
}{
    \subsubsection*{Solution:}
}

\iftoggle{solution}{
    %---------------------------------------------------------------------------------
% QUESTION 2.C
%---------------------------------------------------------------------------------
\secGS{Simulating from Equilibrium}

For this problem we will use the simple coupled tire model where $F_x$ is assumed to be known. Incorporate this into your simulation (We will supply verification code). Run your simulation of the three-state bicycle model for 4 seconds using $\delta$ = -10° and $F_{xr}$ equal to the value you computed in Problem 2B. Use the drift equilibrium found in Problem 1D as the initial condition. Plot $U_y$, $r$, and $U_x$ on the same plot. Does Marty hold the drift? What happens? Did you expect this? If you'd like, visualize using \texttt{animateDrift.m}.

\vspace*{0.5cm}


\expect{Include your plot and an explanation of what you see.}


\iftoggle{condensed}{
    \vspace*{0.5cm}
}{
    \subsubsection*{Solution:}
}

\iftoggle{solution}{
    \input{Solutions/Problem2/problem2c.tex}
}

\iftoggle{template}{
    \begin{solutionorbox}[6in]
    \end{solutionorbox}
    
    \newpage
}

\iftoggle{student}{
%---------------------------------------------------------------------------------
% STUDENT: BEGIN WORK
%---------------------------------------------------------------------------------
% Please box final answer

%---------------------------------------------------------------------------------
% STUDENT: END WORK
%---------------------------------------------------------------------------------
\newpage
}


}

\iftoggle{template}{
    \begin{solutionorbox}[6in]
    \end{solutionorbox}
    
    \newpage
}

\iftoggle{student}{
%---------------------------------------------------------------------------------
% STUDENT: BEGIN WORK
%---------------------------------------------------------------------------------
% Please box final answer

%---------------------------------------------------------------------------------
% STUDENT: END WORK
%---------------------------------------------------------------------------------
\newpage
}


}

\iftoggle{template}{
    \begin{solutionorbox}[6in]
    \end{solutionorbox}
    
    \newpage
}

\iftoggle{student}{
%---------------------------------------------------------------------------------
% STUDENT: BEGIN WORK
%---------------------------------------------------------------------------------
% Please box final answer

%---------------------------------------------------------------------------------
% STUDENT: END WORK
%---------------------------------------------------------------------------------
\newpage
}


%%---------------------------------------------------------------------------------
% QUESTION 2.D
%---------------------------------------------------------------------------------
\secGS{Drifting Intuition}

In problem 2.E, you will implement a longitudinal and lateral controller to stabilize Marty in a drift.  Let's think briefly about the forces at play while drifting in this problem.  In the case of a left-handed drift (e.g. yawrate is positive), you, the driver, have stabilized the vehicle with a negative steering wheel angle.  In each "Observed Behavior" listed in the table, select the correct input to stabilize the vehicle.  When $\delta$ is held constant, select the correct $F_x$ trend to stabilize the vehicle.  When $F_x$ is held constant, select the correct steering input trend to stabilize the vehicle.  Provide an explanation for the input you selected.

\expect{Circle the correct $\delta$ and $F_x$ inputs where applicable.  Provide an explanation for your choice.}


\begin{center}
 \begin{tabular}{|| m{2.1cm} | m{5.5cm} | m{2.5cm} | m{5cm} ||} 
 \hline
 Observed Behavior & $\delta$ & $F_x$ & Explanation\\ [0.5ex] 
 \hline\hline
 $r$ too large & \textbf{more negative} /  \textbf{less negative} & constant &\\& & &\\ 
 \hline
 $r$ too large & constant & \textbf{increase $F_x$} / \textbf{decrease $F_x$} &\\
 \hline
 $|U_y|$ too large & \textbf{more negative} / \textbf{less negative} & constant &\\& & &\\
 \hline
 $|U_y|$ too large & constant & \textbf{increase $F_x$} / \textbf{decrease $F_x$} &\\
 \hline
\end{tabular}
\end{center}

\iftoggle{condensed}{
    \vspace*{0.5cm}
}{
    \subsubsection*{Solution:}
}

\iftoggle{solution}{
    %---------------------------------------------------------------------------------
% QUESTION 2.D
%---------------------------------------------------------------------------------
\secGS{Drifting Intuition}

In problem 2.E, you will implement a longitudinal and lateral controller to stabilize Marty in a drift.  Let's think briefly about the forces at play while drifting in this problem.  In the case of a left-handed drift (e.g. yawrate is positive), you, the driver, have stabilized the vehicle with a negative steering wheel angle.  In each "Observed Behavior" listed in the table, select the correct input to stabilize the vehicle.  When $\delta$ is held constant, select the correct $F_x$ trend to stabilize the vehicle.  When $F_x$ is held constant, select the correct steering input trend to stabilize the vehicle.  Provide an explanation for the input you selected.

\expect{Circle the correct $\delta$ and $F_x$ inputs where applicable.  Provide an explanation for your choice.}


\begin{center}
 \begin{tabular}{|| m{2.1cm} | m{5.5cm} | m{2.5cm} | m{5cm} ||} 
 \hline
 Observed Behavior & $\delta$ & $F_x$ & Explanation\\ [0.5ex] 
 \hline\hline
 $r$ too large & \textbf{more negative} /  \textbf{less negative} & constant &\\& & &\\ 
 \hline
 $r$ too large & constant & \textbf{increase $F_x$} / \textbf{decrease $F_x$} &\\
 \hline
 $|U_y|$ too large & \textbf{more negative} / \textbf{less negative} & constant &\\& & &\\
 \hline
 $|U_y|$ too large & constant & \textbf{increase $F_x$} / \textbf{decrease $F_x$} &\\
 \hline
\end{tabular}
\end{center}

\iftoggle{condensed}{
    \vspace*{0.5cm}
}{
    \subsubsection*{Solution:}
}

\iftoggle{solution}{
    %---------------------------------------------------------------------------------
% QUESTION 2.D
%---------------------------------------------------------------------------------
\secGS{Drifting Intuition}

In problem 2.E, you will implement a longitudinal and lateral controller to stabilize Marty in a drift.  Let's think briefly about the forces at play while drifting in this problem.  In the case of a left-handed drift (e.g. yawrate is positive), you, the driver, have stabilized the vehicle with a negative steering wheel angle.  In each "Observed Behavior" listed in the table, select the correct input to stabilize the vehicle.  When $\delta$ is held constant, select the correct $F_x$ trend to stabilize the vehicle.  When $F_x$ is held constant, select the correct steering input trend to stabilize the vehicle.  Provide an explanation for the input you selected.

\expect{Circle the correct $\delta$ and $F_x$ inputs where applicable.  Provide an explanation for your choice.}


\begin{center}
 \begin{tabular}{|| m{2.1cm} | m{5.5cm} | m{2.5cm} | m{5cm} ||} 
 \hline
 Observed Behavior & $\delta$ & $F_x$ & Explanation\\ [0.5ex] 
 \hline\hline
 $r$ too large & \textbf{more negative} /  \textbf{less negative} & constant &\\& & &\\ 
 \hline
 $r$ too large & constant & \textbf{increase $F_x$} / \textbf{decrease $F_x$} &\\
 \hline
 $|U_y|$ too large & \textbf{more negative} / \textbf{less negative} & constant &\\& & &\\
 \hline
 $|U_y|$ too large & constant & \textbf{increase $F_x$} / \textbf{decrease $F_x$} &\\
 \hline
\end{tabular}
\end{center}

\iftoggle{condensed}{
    \vspace*{0.5cm}
}{
    \subsubsection*{Solution:}
}

\iftoggle{solution}{
    \input{Solutions/Problem2/problem2d.tex}
}

\iftoggle{template}{
    \begin{solutionorbox}[5in]
    \end{solutionorbox}
    
    \newpage
}

\iftoggle{student}{
%---------------------------------------------------------------------------------
% STUDENT: BEGIN WORK
%---------------------------------------------------------------------------------
% Please box final answer

%---------------------------------------------------------------------------------
% STUDENT: END WORK
%---------------------------------------------------------------------------------
\newpage
}


}

\iftoggle{template}{
    \begin{solutionorbox}[5in]
    \end{solutionorbox}
    
    \newpage
}

\iftoggle{student}{
%---------------------------------------------------------------------------------
% STUDENT: BEGIN WORK
%---------------------------------------------------------------------------------
% Please box final answer

%---------------------------------------------------------------------------------
% STUDENT: END WORK
%---------------------------------------------------------------------------------
\newpage
}


}

\iftoggle{template}{
    \begin{solutionorbox}[5in]
    \end{solutionorbox}
    
    \newpage
}

\iftoggle{student}{
%---------------------------------------------------------------------------------
% STUDENT: BEGIN WORK
%---------------------------------------------------------------------------------
% Please box final answer

%---------------------------------------------------------------------------------
% STUDENT: END WORK
%---------------------------------------------------------------------------------
\newpage
}


%%---------------------------------------------------------------------------------
% QUESTION 2.E
%---------------------------------------------------------------------------------
\secGS{Controlled Drifting}

To sustain the drift let’s add feedback terms to the values of $\delta$ and $F_{xr}$. Use a simple longitudinal controlled to track the desired longitudinal speed:
$$F_{xr}=K_x(U_{X,\text{des}}-U_x)+F_{x,ff}$$
Where $K_x$ = 2,000 N/(m/s), $U_{x,\text{des}}$ = 8 m/s, and $F_{x,ff}$ is the value you found in Problem 2B. For the feedback steering, using proportional control on $U_y$ and $r$:
$$\delta=k_r(r_{\text{des}}-r)+k_y(U_{y,\text{des}}-U_y)+\delta_{ff}$$
The absolute value $k_r$ = 1 s. The absolute value of $k_y$ = 0.5 rad/(m/s), and $\delta_{ff}$ = -10°. Based on your observations in Problem 2.D, select the sign for $k_r$ and $k_y$.  Simulate for 9 seconds using the drift equilibrium as the initial condition. Plot $U_y$, $r$, and $U_x$ and visualize the animation. Are we drifting now?  What is the steady state sideslip angle?

\expect{Plot of the states with an explanation of whether we are drifting and why. A calculation for the steady state sideslip angle.}

\iftoggle{condensed}{
    \vspace*{0.5cm}
}{
    \subsubsection*{Solution:}
}

\iftoggle{solution}{
    %---------------------------------------------------------------------------------
% QUESTION 2.E
%---------------------------------------------------------------------------------
\secGS{Controlled Drifting}

To sustain the drift let’s add feedback terms to the values of $\delta$ and $F_{xr}$. Use a simple longitudinal controlled to track the desired longitudinal speed:
$$F_{xr}=K_x(U_{X,\text{des}}-U_x)+F_{x,ff}$$
Where $K_x$ = 2,000 N/(m/s), $U_{x,\text{des}}$ = 8 m/s, and $F_{x,ff}$ is the value you found in Problem 2B. For the feedback steering, using proportional control on $U_y$ and $r$:
$$\delta=k_r(r_{\text{des}}-r)+k_y(U_{y,\text{des}}-U_y)+\delta_{ff}$$
The absolute value $k_r$ = 1 s. The absolute value of $k_y$ = 0.5 rad/(m/s), and $\delta_{ff}$ = -10°. Based on your observations in Problem 2.D, select the sign for $k_r$ and $k_y$.  Simulate for 9 seconds using the drift equilibrium as the initial condition. Plot $U_y$, $r$, and $U_x$ and visualize the animation. Are we drifting now?  What is the steady state sideslip angle?

\expect{Plot of the states with an explanation of whether we are drifting and why. A calculation for the steady state sideslip angle.}

\iftoggle{condensed}{
    \vspace*{0.5cm}
}{
    \subsubsection*{Solution:}
}

\iftoggle{solution}{
    %---------------------------------------------------------------------------------
% QUESTION 2.E
%---------------------------------------------------------------------------------
\secGS{Controlled Drifting}

To sustain the drift let’s add feedback terms to the values of $\delta$ and $F_{xr}$. Use a simple longitudinal controlled to track the desired longitudinal speed:
$$F_{xr}=K_x(U_{X,\text{des}}-U_x)+F_{x,ff}$$
Where $K_x$ = 2,000 N/(m/s), $U_{x,\text{des}}$ = 8 m/s, and $F_{x,ff}$ is the value you found in Problem 2B. For the feedback steering, using proportional control on $U_y$ and $r$:
$$\delta=k_r(r_{\text{des}}-r)+k_y(U_{y,\text{des}}-U_y)+\delta_{ff}$$
The absolute value $k_r$ = 1 s. The absolute value of $k_y$ = 0.5 rad/(m/s), and $\delta_{ff}$ = -10°. Based on your observations in Problem 2.D, select the sign for $k_r$ and $k_y$.  Simulate for 9 seconds using the drift equilibrium as the initial condition. Plot $U_y$, $r$, and $U_x$ and visualize the animation. Are we drifting now?  What is the steady state sideslip angle?

\expect{Plot of the states with an explanation of whether we are drifting and why. A calculation for the steady state sideslip angle.}

\iftoggle{condensed}{
    \vspace*{0.5cm}
}{
    \subsubsection*{Solution:}
}

\iftoggle{solution}{
    \input{Solutions/Problem2/problem2e.tex}
}

\iftoggle{template}{
    \begin{solutionorbox}[5in]
    \end{solutionorbox}
    
    \newpage
}

\iftoggle{student}{
%---------------------------------------------------------------------------------
% STUDENT: BEGIN WORK
%---------------------------------------------------------------------------------
% Please box final answer

%---------------------------------------------------------------------------------
% STUDENT: END WORK
%---------------------------------------------------------------------------------
\newpage
}


}

\iftoggle{template}{
    \begin{solutionorbox}[5in]
    \end{solutionorbox}
    
    \newpage
}

\iftoggle{student}{
%---------------------------------------------------------------------------------
% STUDENT: BEGIN WORK
%---------------------------------------------------------------------------------
% Please box final answer

%---------------------------------------------------------------------------------
% STUDENT: END WORK
%---------------------------------------------------------------------------------
\newpage
}


}

\iftoggle{template}{
    \begin{solutionorbox}[5in]
    \end{solutionorbox}
    
    \newpage
}

\iftoggle{student}{
%---------------------------------------------------------------------------------
% STUDENT: BEGIN WORK
%---------------------------------------------------------------------------------
% Please box final answer

%---------------------------------------------------------------------------------
% STUDENT: END WORK
%---------------------------------------------------------------------------------
\newpage
}


\newpage

%---------------------------------------------------------------------------------
% PROBLEM 3
%---------------------------------------------------------------------------------
% Problem Intro
%\secPr{Evaluating our Models Against Experimental Data}

Before plunging deeper into the development of vehicle dynamics models, we should see how well these simple models
predict the measurements we can obtain from a vehicle. We have some data of Niki doing a double lane change maneuver in
the lot off Searsville road on Stanford's campus at various speeds. A high-precision GPS/INS system in the car gives us
accurate measurements of lateral velocity, yaw rate, lateral acceleration, and vehicle speed. We also measure steer
angle through the actuators built into the car. In this problem, we're going to compare these measurements to what we
predict using the models developed in class. While there are a lot of simplifications involved, you should find the
results at least somewhat impressive. If not, chances are you have a bug somewhere...

Three experiments were run, one at low speed, one at intermediate speed, and one at high speed. We have separated all of
the data into MATLAB cell arrays, with the first cell corresponding to the low speed test, etc. 

The following \GRno{} questions will walk you through modifying your simulations so that they can use recorded steer
angle and longitudinal velocity as inputs. With this modification made, you can now compare the experimental data to
three different models.

\textit{NOTE: Though we have written our simulator to execute at 1kHz, our data logger records data on Niki at 200Hz.  In research and industry, we frequently need to resample data that has been logged. One way to address this here is to post-process our recorded data early in our script by creating two new vectors of steering angle and velocity that correspond in time to our simulation time vector. 
You can use the MATLAB function} \verb!interp1! \textit{along with the vector of simulation time to interpolate a vector of
steering angles and velocity from the recorded data. We have implemented this for you in \GRno{}, but it's a great trick to know if
you work with experimental data in the future.}

\vspace*{1cm}


% Student Prompts and Responses
%\input{Template/Problem3/problem3a_new.tex}
%input{Template/Problem3/problem3b.tex}
%%---------------------------------------------------------------------------------
% PROBLEM 3 - PART C
%---------------------------------------------------------------------------------
\secGS{Lookahead vs PD Control}
Show that for a constant vehicle speed, the lookahead controller is equivalent to a PD controller of the form:

\al{
\delta = K_p e + K_d \frac{de}{dt}
}

Calculate the parameters $K_p$ and $K_d$ in terms of $K_{la}$ and $d_{la}$. You should see that the lookahead control law provides a simple way of scaling, or gain scheduling, proportional and derivative gains as the vehicle speed changes.

\vspace*{0.5cm}

\expect{Write the parameters $K_{p}$ and $K_{d}$ in terms of $K_{la}$ and $d_{la}$.}

\iftoggle{condensed}{
    \vspace*{0.5cm}
}{
    \subsubsection*{Solution:}
}

\iftoggle{solution}{
    %---------------------------------------------------------------------------------
% PROBLEM 3 - PART C
%---------------------------------------------------------------------------------
\secGS{Lookahead vs PD Control}
Show that for a constant vehicle speed, the lookahead controller is equivalent to a PD controller of the form:

\al{
\delta = K_p e + K_d \frac{de}{dt}
}

Calculate the parameters $K_p$ and $K_d$ in terms of $K_{la}$ and $d_{la}$. You should see that the lookahead control law provides a simple way of scaling, or gain scheduling, proportional and derivative gains as the vehicle speed changes.

\vspace*{0.5cm}

\expect{Write the parameters $K_{p}$ and $K_{d}$ in terms of $K_{la}$ and $d_{la}$.}

\iftoggle{condensed}{
    \vspace*{0.5cm}
}{
    \subsubsection*{Solution:}
}

\iftoggle{solution}{
    %---------------------------------------------------------------------------------
% PROBLEM 3 - PART C
%---------------------------------------------------------------------------------
\secGS{Lookahead vs PD Control}
Show that for a constant vehicle speed, the lookahead controller is equivalent to a PD controller of the form:

\al{
\delta = K_p e + K_d \frac{de}{dt}
}

Calculate the parameters $K_p$ and $K_d$ in terms of $K_{la}$ and $d_{la}$. You should see that the lookahead control law provides a simple way of scaling, or gain scheduling, proportional and derivative gains as the vehicle speed changes.

\vspace*{0.5cm}

\expect{Write the parameters $K_{p}$ and $K_{d}$ in terms of $K_{la}$ and $d_{la}$.}

\iftoggle{condensed}{
    \vspace*{0.5cm}
}{
    \subsubsection*{Solution:}
}

\iftoggle{solution}{
    \input{Solutions/Problem3/problem3c.tex}
    \newpage
}


\iftoggle{template}{
    \vspace*{12cm}
    \begin{solutionorbox}[1in]
    \end{solutionorbox}
    \newpage
}

\iftoggle{student}{
%---------------------------------------------------------------------------------
% STUDENT: BEGIN WORK
%---------------------------------------------------------------------------------


% Please box final answer

%---------------------------------------------------------------------------------
% STUDENT: END WORK
%---------------------------------------------------------------------------------
    \newpage
}


    \newpage
}


\iftoggle{template}{
    \vspace*{12cm}
    \begin{solutionorbox}[1in]
    \end{solutionorbox}
    \newpage
}

\iftoggle{student}{
%---------------------------------------------------------------------------------
% STUDENT: BEGIN WORK
%---------------------------------------------------------------------------------


% Please box final answer

%---------------------------------------------------------------------------------
% STUDENT: END WORK
%---------------------------------------------------------------------------------
    \newpage
}


    \newpage
}


\iftoggle{template}{
    \vspace*{12cm}
    \begin{solutionorbox}[1in]
    \end{solutionorbox}
    \newpage
}

\iftoggle{student}{
%---------------------------------------------------------------------------------
% STUDENT: BEGIN WORK
%---------------------------------------------------------------------------------


% Please box final answer

%---------------------------------------------------------------------------------
% STUDENT: END WORK
%---------------------------------------------------------------------------------
    \newpage
}


%\newpage
%---------------------------------------------------------------------------------
% REMARKS
%---------------------------------------------------------------------------------
\iftoggle{solution}{
    \input{Solutions/remarks.tex}
    \newpage
}

%---------------------------------------------------------------------------------
% APPENDICES
%---------------------------------------------------------------------------------
\appendix


\end{document}
