%---------------------------------------------------------------------------------
% QUESTION 1.A
%---------------------------------------------------------------------------------
\secGR{Linearization of our Velocity Dynamics}


Up until now, we have linearized our dynamics via linear \textit{assumptions}, such as small angle approximations and linear tires. We will now directly linearize our dynamics (which allows analysis and control of cool vehicle behavior such as drifting). 

Consider just our velocity states, so $x = [r, U_y]^T$. We can write our dynamics as being a nonlinear function of our state and steering input (forget $F_{xr}$ for now - assume we are using the same longitudinal controller from HW5):
$$ \dot{x} = f(x,\delta)$$
We can linearize the dynamics and write our system as: 
$$\dot{x} = f(x_{eq}) + A(x-x_{eq}) + B(\delta-\delta_{eq})= A(x-x_{eq}) + B(\delta-\delta_{eq})$$
$$ \dot{\bar{x}} = A\bar{x} + B\bar{\delta} $$


\begin{equation}
A = 
\begin{bmatrix}
\frac{\partial \dot r}{\partial r}  & \frac{\partial \dot r}{\partial U_y} \\
\frac{\partial \dot U_y}{\partial r}  & \frac{\partial \dot U_y}{\partial U_y}\\
\end{bmatrix}_{x=x_{eq}} 
\end{equation}

\begin{equation}
B = \begin{bmatrix}
\frac{\partial \dot r}{\partial \delta} \\
\frac{\partial \dot U_y}{\partial \delta}\\
\end{bmatrix}_{x=x_{eq}} 
\end{equation}

To calculate the partial derivatives, we need to carefully consider what each term in our dynamics is a function of. Based on the Fiala lateral tire model:
$$F_{yf} = f(\alpha_f)=f(U_y, r, \delta)$$
Generally while drifting, $\alpha_r > \alpha_{sl}$ so:
$$F_{yr} = f(F_{xr})$$

So to calculate the first entry in the A matrix:
$$ \frac{\partial \dot r}{\partial r} \biggr\rvert_{x=x_{eq}} = \frac{1}{I_z}(a\frac{\partial F_{yf}}{\partial r}cos(\delta)-b\frac{\partial F_{yr}}{\partial r}) \biggr\rvert_{x=x_{eq}} = \frac{acos(\delta)}{I_z}\frac{\partial F_{yf}}{\partial r} \biggr\rvert_{x=x_{eq}}$$

To calculate $\frac{\partial F_{yf}}{\partial r}$, we then need to take the derivative of each term in the Fiala model with respect to $r$. The rest of the derivatives can be computed similarly.

For this two state model we have computed the A \& B matrices for you (so you won't need to take the derivative of the Fiala model). However, make sure you understand how one could compute each term in A \& B, as you will be finding some of the derivatives in the next problem.


\begin{equation}
A = \begin{bmatrix} 
-1.1963 & -0.744 \\
-11.5025 & -0.7339\\
\end{bmatrix}
\end{equation}
\begin{equation}
B = \begin{bmatrix} 
13.1451\\
12.967\\
\end{bmatrix}
\end{equation}

Finally, lets use the control law $\bar{\delta} = -K\bar{x}$ (full state feedback).

\textbf{Find the gain vector K that puts the poles of the closed loop system at -4 and -8. Compute this by hand and show your work. Submit your gain vector to MATLAB Grader.}




\iftoggle{condensed}{
    \vspace*{0.5cm}
}{
    \subsubsection*{Solution:}
}

\iftoggle{solution}{
    %---------------------------------------------------------------------------------
% QUESTION 1.A
%---------------------------------------------------------------------------------
\secGR{Determining an Appropriate Lookahead Gain}

Let's start with the "lookdown" controller ($x_{la} = 0$). Our first step is to determine a gain that seems physically
reasonable.

Follow the prompts in \GRno{} to create a script to calculate $K_{la}$ such that our controller produces a steer 
angle of 3 degrees when the lateral error is one meter and the vehicle is pointed straight along the path.

You are to modify the lookahead gain, \verb!K_la!, to the appropriate value.

\vspace*{0.5cm}

\iftoggle{solution}{
    \input{Solutions/matlabgrader.tex}
}


}

\iftoggle{template}{
    \begin{solutionorbox}[5in]
    \end{solutionorbox}
    
    \newpage
}

\iftoggle{student}{
%---------------------------------------------------------------------------------
% STUDENT: BEGIN WORK
%---------------------------------------------------------------------------------
% Please box final answer

%---------------------------------------------------------------------------------
% STUDENT: END WORK
%---------------------------------------------------------------------------------
\newpage
}

