%---------------------------------------------------------------------------------
% QUESTION 2.E
%---------------------------------------------------------------------------------
\secGS{Controlled Drifting}

To sustain the drift let’s add feedback terms to the values of $\delta$ and $F_{xr}$. Use a simple longitudinal controlled to track the desired longitudinal speed:
$$F_{xr}=K_x(U_{X,\text{des}}-U_x)+F_{x,ff}$$
Where $K_x$ = 2,000 N/(m/s), $U_{x,\text{des}}$ = 8 m/s, and $F_{x,ff}$ is the value you found in Problem 2B. For the feedback steering, using proportional control on $U_y$ and $r$:
$$\delta=k_r(r_{\text{des}}-r)+k_y(U_{y,\text{des}}-U_y)+\delta_{ff}$$
The absolute value $k_r$ = 1 s. The absolute value of $k_y$ = 0.5 rad/(m/s), and $\delta_{ff}$ = -10°. Based on your observations in Problem 2.D, select the sign for $k_r$ and $k_y$.  Simulate for 9 seconds using the drift equilibrium as the initial condition. Plot $U_y$, $r$, and $U_x$ and visualize the animation. Are we drifting now?  What is the steady state sideslip angle?

\expect{Plot of the states with an explanation of whether we are drifting and why. A calculation for the steady state sideslip angle.}

\iftoggle{condensed}{
    \vspace*{0.5cm}
}{
    \subsubsection*{Solution:}
}

\iftoggle{solution}{
    %---------------------------------------------------------------------------------
% QUESTION 2.E
%---------------------------------------------------------------------------------
\secGS{Controlled Drifting}

To sustain the drift let’s add feedback terms to the values of $\delta$ and $F_{xr}$. Use a simple longitudinal controlled to track the desired longitudinal speed:
$$F_{xr}=K_x(U_{X,\text{des}}-U_x)+F_{x,ff}$$
Where $K_x$ = 2,000 N/(m/s), $U_{x,\text{des}}$ = 8 m/s, and $F_{x,ff}$ is the value you found in Problem 2B. For the feedback steering, using proportional control on $U_y$ and $r$:
$$\delta=k_r(r_{\text{des}}-r)+k_y(U_{y,\text{des}}-U_y)+\delta_{ff}$$
The absolute value $k_r$ = 1 s. The absolute value of $k_y$ = 0.5 rad/(m/s), and $\delta_{ff}$ = -10°. Based on your observations in Problem 2.D, select the sign for $k_r$ and $k_y$.  Simulate for 9 seconds using the drift equilibrium as the initial condition. Plot $U_y$, $r$, and $U_x$ and visualize the animation. Are we drifting now?  What is the steady state sideslip angle?

\expect{Plot of the states with an explanation of whether we are drifting and why. A calculation for the steady state sideslip angle.}

\iftoggle{condensed}{
    \vspace*{0.5cm}
}{
    \subsubsection*{Solution:}
}

\iftoggle{solution}{
    %---------------------------------------------------------------------------------
% QUESTION 2.E
%---------------------------------------------------------------------------------
\secGS{Controlled Drifting}

To sustain the drift let’s add feedback terms to the values of $\delta$ and $F_{xr}$. Use a simple longitudinal controlled to track the desired longitudinal speed:
$$F_{xr}=K_x(U_{X,\text{des}}-U_x)+F_{x,ff}$$
Where $K_x$ = 2,000 N/(m/s), $U_{x,\text{des}}$ = 8 m/s, and $F_{x,ff}$ is the value you found in Problem 2B. For the feedback steering, using proportional control on $U_y$ and $r$:
$$\delta=k_r(r_{\text{des}}-r)+k_y(U_{y,\text{des}}-U_y)+\delta_{ff}$$
The absolute value $k_r$ = 1 s. The absolute value of $k_y$ = 0.5 rad/(m/s), and $\delta_{ff}$ = -10°. Based on your observations in Problem 2.D, select the sign for $k_r$ and $k_y$.  Simulate for 9 seconds using the drift equilibrium as the initial condition. Plot $U_y$, $r$, and $U_x$ and visualize the animation. Are we drifting now?  What is the steady state sideslip angle?

\expect{Plot of the states with an explanation of whether we are drifting and why. A calculation for the steady state sideslip angle.}

\iftoggle{condensed}{
    \vspace*{0.5cm}
}{
    \subsubsection*{Solution:}
}

\iftoggle{solution}{
    %---------------------------------------------------------------------------------
% QUESTION 2.E
%---------------------------------------------------------------------------------
\secGS{Controlled Drifting}

To sustain the drift let’s add feedback terms to the values of $\delta$ and $F_{xr}$. Use a simple longitudinal controlled to track the desired longitudinal speed:
$$F_{xr}=K_x(U_{X,\text{des}}-U_x)+F_{x,ff}$$
Where $K_x$ = 2,000 N/(m/s), $U_{x,\text{des}}$ = 8 m/s, and $F_{x,ff}$ is the value you found in Problem 2B. For the feedback steering, using proportional control on $U_y$ and $r$:
$$\delta=k_r(r_{\text{des}}-r)+k_y(U_{y,\text{des}}-U_y)+\delta_{ff}$$
The absolute value $k_r$ = 1 s. The absolute value of $k_y$ = 0.5 rad/(m/s), and $\delta_{ff}$ = -10°. Based on your observations in Problem 2.D, select the sign for $k_r$ and $k_y$.  Simulate for 9 seconds using the drift equilibrium as the initial condition. Plot $U_y$, $r$, and $U_x$ and visualize the animation. Are we drifting now?  What is the steady state sideslip angle?

\expect{Plot of the states with an explanation of whether we are drifting and why. A calculation for the steady state sideslip angle.}

\iftoggle{condensed}{
    \vspace*{0.5cm}
}{
    \subsubsection*{Solution:}
}

\iftoggle{solution}{
    \input{Solutions/Problem2/problem2e.tex}
}

\iftoggle{template}{
    \begin{solutionorbox}[5in]
    \end{solutionorbox}
    
    \newpage
}

\iftoggle{student}{
%---------------------------------------------------------------------------------
% STUDENT: BEGIN WORK
%---------------------------------------------------------------------------------
% Please box final answer

%---------------------------------------------------------------------------------
% STUDENT: END WORK
%---------------------------------------------------------------------------------
\newpage
}


}

\iftoggle{template}{
    \begin{solutionorbox}[5in]
    \end{solutionorbox}
    
    \newpage
}

\iftoggle{student}{
%---------------------------------------------------------------------------------
% STUDENT: BEGIN WORK
%---------------------------------------------------------------------------------
% Please box final answer

%---------------------------------------------------------------------------------
% STUDENT: END WORK
%---------------------------------------------------------------------------------
\newpage
}


}

\iftoggle{template}{
    \begin{solutionorbox}[5in]
    \end{solutionorbox}
    
    \newpage
}

\iftoggle{student}{
%---------------------------------------------------------------------------------
% STUDENT: BEGIN WORK
%---------------------------------------------------------------------------------
% Please box final answer

%---------------------------------------------------------------------------------
% STUDENT: END WORK
%---------------------------------------------------------------------------------
\newpage
}


}

\iftoggle{template}{
    \begin{solutionorbox}[5in]
    \end{solutionorbox}
    
    \newpage
}

\iftoggle{student}{
%---------------------------------------------------------------------------------
% STUDENT: BEGIN WORK
%---------------------------------------------------------------------------------
% Please box final answer

%---------------------------------------------------------------------------------
% STUDENT: END WORK
%---------------------------------------------------------------------------------
\newpage
}

