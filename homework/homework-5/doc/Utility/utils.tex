
% This is a borrowed LaTeX template file for lecture notes for CS267,
% Applications of Parallel Computing, UCBerkeley EECS Department.
% Now being used for CMU's 10725 Fall 2012 Optimization course
% taught by Geoff Gordon and Ryan Tibshirani.  When preparing 
% LaTeX notes for this class, please use this template.
%
% To familiarize yourself with this template, the body contains
% some examples of its use.  Look them over.  Then you can
% run LaTeX on this file.  After you have LaTeXed this file then
% you can look over the result either by printing it out with
% dvips or using xdvi. "pdflatex template.tex" should also work.

\setlength{\oddsidemargin}{0 in}
\setlength{\evensidemargin}{0 in}
\setlength{\topmargin}{-0.6 in}
\setlength{\textwidth}{6.5 in}
\setlength{\textheight}{8.5 in}
\setlength{\headsep}{0.75 in}
\setlength{\parindent}{0 in}
\setlength{\parskip}{0.1 in}
\extrafootheight{-0.5in}

% flags to display template, solution, or condensed versions
% toggletrue{variable} or togglefalse{variable}
% nothing stopping you from turning all 3 on and getting a 
% super long document
\newtoggle{template}
\newtoggle{solution}
\newtoggle{condensed}
\newtoggle{student}

\newcommand{\template}{
    \toggletrue{template}
    \togglefalse{solution}
    \togglefalse{condensed}
    \togglefalse{student}
}

\newcommand{\solutions}{
    \toggletrue{solution}
    \togglefalse{template}
    \togglefalse{condensed}
    \togglefalse{student}
}

\newcommand{\condensed}{
    \toggletrue{condensed}
    \togglefalse{template}
    \togglefalse{solution}
    \togglefalse{student}
}

\newcommand{\student}{
    \toggletrue{student}
    \togglefalse{template}
    \togglefalse{solution}
    \togglefalse{condensed}
}

%% For hw6
%\newcommand*\circled[1]{\tikz[baseline=(char.base)]{
%    \node[shape=circle,draw,inner sep=2pt] (char) {#1};}}
%
%% The following commands set up the lecnum (lecture number)
%% counter and make various numbering schemes work relative
%% to the lecture number.

\newcounter{lecnum}

%\renewcommand{\thepage}{\thelecnum.\arabic{page}}
%\renewcommand{\thesection}{\thelecnum.\arabic{section}}
\renewcommand{\thesubsection}{\arabic{section}.\Alph{subsection}}
\renewcommand{\theequation}{\thelecnum.\arabic{equation}}
\renewcommand{\thefigure}{\thelecnum.\arabic{figure}}
\renewcommand{\thetable}{\thelecnum.\arabic{table}}

%\newcommand{\cc}{{\circ}}
%\newcommand{\Cf}{{\tilde{C}_{f\cc}}}
%\newcommand{\Cr}{{\tilde{C}_{r\cc}}}
%\newcommand{\Kpf}{{K_{\phi f}}}
%\newcommand{\Kpr}{{K_{\phi r}}}
%\newcommand{\Kp}{{K_{\phi}}}
%\newcommand{\cs}{{c_\mathrm{sky}}}


%
% The following macro is used to generate the header.
%
\newcommand{\lecture}[4]{
   \pagestyle{myheadings}
   \thispagestyle{plain}
   \newpage
   \setcounter{lecnum}{#1}
   \setcounter{page}{1}
   \noindent
   \begin{center}
   \framebox{
      \vbox{\vspace{2mm}
    \hbox to 6.28in { {\bf ME227: Vehicle Dynamics and Control
	\hfill Spring 2021} }
       \vspace{4mm}
       \hbox to 6.28in { {\Large \hfill Lecture #1: #2  \hfill} }
       \vspace{2mm}
       \hbox to 6.28in { {#3 \hfill \textcopyright #4 } }
      \vspace{2mm}}
   }
   \end{center}
   \markboth{Lecture #1: #2}{Lecture #1: #2}

}

\newcommand{\assignment}[4]{
   \pagestyle{myheadings}
   \thispagestyle{plain}
   \newpage
   \setcounter{lecnum}{#1}
   \setcounter{page}{1}
   \noindent
   \begin{center}
   \framebox{
      \vbox{\vspace{2mm}
    \hbox to 6.28in { {\bf ME227: Vehicle Dynamics and Control
	\hfill Spring 2021} }
       \vspace{4mm}
       \hbox to 6.28in { {\Large \hfill Assignment #1: #2  \hfill} }
       \vspace{2mm}
       \hbox to 6.28in { {#3 \hfill \textcopyright #4 } }
      \vspace{2mm}}
   }
   \end{center}
   \markboth{Assignment #1: #2}{Assignment #1: #2}
}

\newcommand{\solutionME}[4]{
   \pagestyle{myheadings}
   \thispagestyle{plain}
   \newpage
   \setcounter{lecnum}{#1}
   \setcounter{page}{1}
   \noindent
   \begin{center}
   \framebox{
      \vbox{\vspace{2mm}
    \hbox to 6.28in { {\bf ME227: Vehicle Dynamics and Control
	\hfill Spring 2017} }
       \vspace{4mm}
       \hbox to 6.28in { {\Large \hfill Solution #1: #2  \hfill} }
       \vspace{2mm}
       \hbox to 6.28in { {#3 \hfill \textcopyright #4 } }
      \vspace{2mm}}
   }
   \end{center}
   \markboth{Solution #1: #2}{Solution #1: #2}
}

\newcommand{\project}[4]{
   \pagestyle{myheadings}
   \thispagestyle{plain}
   \newpage
   \setcounter{lecnum}{#1}
   \setcounter{page}{1}
   \noindent
   \begin{center}
   \framebox{
      \vbox{\vspace{2mm}
    \hbox to 6.28in { {\bf ME227: Vehicle Dynamics and Control
	\hfill Spring 2021} }
       \vspace{4mm}
       \hbox to 6.28in { {\Large \hfill Project: #2  \hfill} }
       \vspace{2mm}
       \hbox to 6.28in { {#3 \hfill \textcopyright #4 } }
      \vspace{2mm}}
   }
   \end{center}
   \markboth{Project: #2}{Project: #2}
}

%
% Convention for citations is authors' initials followed by the year.
% For example, to cite a paper by Leighton and Maggs you would type
% \cite{LM89}, and to cite a paper by Strassen you would type \cite{S69}.
% (To avoid bibliography problems, for now we redefine the \cite command.)
% Also commands that create a suitable format for the reference list.
\renewcommand{\cite}[1]{[#1]}
\def\beginrefs{\begin{list}%
        {[\arabic{equation}]}{\usecounter{equation}
         \setlength{\leftmargin}{2.0truecm}\setlength{\labelsep}{0.4truecm}%
         \setlength{\labelwidth}{1.6truecm}}}
\def\endrefs{\end{list}}
\def\bibentry#1{\item[\hbox{[#1]}]}

%Use this command for a figure; it puts a figure in wherever you want it.
%usage: \fig{NUMBER}{SPACE-IN-INCHES}{CAPTION}

% Use these for theorems, lemmas, proofs, etc.
\newtheorem{theorem}{Theorem}[lecnum]
\newtheorem{lemma}[theorem]{Lemma}
\newtheorem{proposition}[theorem]{Proposition}
\newtheorem{claim}[theorem]{Claim}
\newtheorem{corollary}[theorem]{Corollary}
\newtheorem{definition}[theorem]{Definition}
\newenvironment{proof}{{\bf Proof:}}{\hfill\rule{2mm}{2mm}}

% **** IF YOU WANT TO DEFINE ADDITIONAL MACROS FOR YOURSELF, PUT THEM HERE:

\newcommand\E{\mathbb{E}}

\newcommand{\eX}{{\boldsymbol{\hat{e}_X}}}
\newcommand{\eY}{{\boldsymbol{\hat{e}_Y}}}
\newcommand{\eZ}{{\boldsymbol{\hat{e}_Z}}}
\newcommand{\pVE}{{\psi_{VE}}}
\newcommand{\pVEdot}{\dot{\psi}_{VE}}

%\newcommand{\roa}{\uvec{r}_{oa}}

% basis vector: arg1 is name
\newcommand{\basvec}[1]{\boldsymbol{\hat{e}_{#1}}}

% "radius" vector (position vector named r) from arg1 to arg2

\renewcommand{\vec}[1]{%
  \smash{\ensurestackMath{\stackengine{1pt}{#1}{\scriptscriptstyle\sim}{U}{c}{F}{F}{S}}}
  \vphantom{#1}
}
\newcommand{\roa}{\vec{r}_{oa}}
\newcommand{\rv}[3]{\vec{r}_{#1#2,#3}}


% similar with velocity and acceleration vectors
\newcommand{\vv}[3]{\vec{v}_{#1#2,#3}}
\newcommand{\wv}[3]{\vec{\omega}_{#1#2,#3}}
\newcommand{\av}[3]{\vec{a}_{#1#2,#3}}

% vector derivatives
\newcommand{\drv}[3]{\vec{\dot{r}}_{#1#2,#3}}
\newcommand{\dvv}[3]{\vec{\dot{v}}_{#1#2,#3}}
\newcommand{\dwv}[3]{\vec{\dot{\omega}}_{#1#2,#3}}

% forces
\newcommand{\F}[2]{\text{F}_\text{#1#2}}

% cornering stiffnesses and slip angles
\newcommand{\Caf}{C_{\alpha f}}
\newcommand{\Car}{C_{\alpha r}}
\newcommand{\af}{\alpha f}
\newcommand{\ar}{\alpha r}

% vehicle states and derivatives
\newcommand{\Uxdot}{\dot{U}_x}
\newcommand{\Uydot}{\dot{U}_y}
\newcommand{\rdot}{\dot{r}}
\newcommand{\dpsi}{\Delta\Psi}
\newcommand{\dpsidot}{\Delta\dot{\Psi}}
\newcommand{\edot}{\dot{e}}

% Gains
\newcommand{\Kla}{K_{la}}
\newcommand{\xla}{x_{la}}

% Commands for Gradescope and Grader
\newcommand{\GS}{\textbf{Gradescope}}
\newcommand{\GR}{\textbf{MATLAB Grader}}
\newcommand{\MO}{\textbf{MATLAB Online}}
\newcommand{\GSno}{Gradescope}
\newcommand{\GRno}{MATLAB Grader}
\newcommand{\MOno}{MATLAB Online}

% Commands for numberless section headings while incrementing counter
\newcommand{\secNoNum}[1]{\refstepcounter{section}\section*{#1}}
\newcommand{\subsecNoNum}[1]{\refstepcounter{subsection}\subsection*{#1}}
\newcommand{\ph}{Problem \thesection}
\newcommand{\qh}{Question \thesection.\Alph{subsection}}
\newcommand{\apph}{Appendix \Alph{section}}

% Commands for questions subsection headings
\newcommand{\secPr}[1]{%
    \secNoNum{%
        \ph{} -- #1
    }
    \label{sec:#1}
}

\newcommand{\secApp}[1]{%
    \secNoNum{%
        \apph{} -- #1
    }
    \label{appendix:#1}
}

\newcommand{\secGR}[1]{%
    \subsecNoNum{%
        \color{blue}
        \qh{} -- #1 (\GRno{})
    }
    \label{subsecGR:#1}
}

\newcommand{\secMOGR}[1]{%
    \subsecNoNum{%
        \color{blue}
        \qh{} -- #1 (\MOno{} / \GRno{})
    }
    \label{subsecMOGR:#1}
}

\newcommand{\secMOGS}[1]{%
    \subsecNoNum{%
        \qh{} -- #1 (\MOno{} / \GSno{})
    }
    \label{subsecMOGS:#1}
}

\newcommand{\secGS}[1]{%
    \subsecNoNum{%
        \qh{} -- #1 (\GSno{})
    }
    \label{subsecGS:#1}
}

% Reference other question subsections
\newcommand{\qrefGR}[1]{%
    \textbf{%
        Question \ref{subsecGR:#1}%
    }%
}

\newcommand{\qrefGS}[1]{%
    \textbf{%
        Question \ref{subsecGS:#1}%
    }%
}

\newcommand{\qrefMOGR}[1]{%
    \textbf{%
        Question \ref{subsecMOGR:#1}%
    }%
}

\newcommand{\qrefMOGS}[1]{%
    \textbf{%
        Question \ref{subsecMOGS:#1}%
    }%
}

\newcommand{\pref}[1]{%
    \textbf{%
        Problem \ref{sec:#1}%
    }%
}


% Shortcut for equation
\newcommand{\eq}[1]{%
    \begin{equation*}
        #1
    \end{equation*}
}

% Shortcut for align
\newcommand{\al}[1]{%
    \begin{align*}
        #1
    \end{align*}
}

% Shortcut for problem expectation
\newcommand{\expect}[1]{%
    \textbf{%
        \textit{%
            #1%
        }%
    }%
}

% Shortcut for bold
\newcommand{\bld}[1]{\textbf{#1}}

% Change solution title
\renewcommand{\solutiontitle}{\noindent\enspace}

% Matlab code helper macros
\newcommand{\matin}[1]{\lstinline[style=Matlab-editor]{#1}}
\newcommand{\matld}[1]{\lstinputlisting[style=Matlab-editor, basicstyle=\footnotesize]{#1}}

% Remove question numbers
\qformat{}
\renewcommand{\questionshook}{%
\setlength{\leftmargin}{0pt}%
\setlength{\labelwidth}{-\labelsep}%
}

% Fix duplicate question labeling
\makeatletter
\xpatchcmd{\questions}
  {question@\arabic{question}}
  {question@\arabic{page}@\arabic{question}}
  {}{}
\makeatother

% Shortcut for underline and bold
\newcommand{\ubf}[1]{\underline{\textbf{#1}}}

