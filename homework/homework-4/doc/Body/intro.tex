\assignment{4}{Lanekeeping}{Due Thursday, April 29 at 5:00pm PST}{Chris Gerdes, 2017}

\hspace{0.5cm}

\section*{Purpose}

In this assignment, we will look at the characteristics of the simple lookahead controller we developed in class. We will also put together a simulation that will serve as our model for testing controllers prior to implementing them on Niki in your project teams.

\section*{Instructions}

This and following homework assignments will be submitted using two different tools, \GSno{} and \GRno{}.
Throughout the assignment, each prompt will be marked with either \GS{} or \GR{} to make it clear
where you should be submitting the answer to that problem. \GRno{} questions will also be shown in
\textbf{\color{blue}Blue} to make them easy to see.

All written portions must be turned in through Gradescope. See the Piazza post on homework guidelines for more
instructions on the different homework resources available to you. Whatever format you decide to use, please \fbox{\textbf{BOX}} all of your final answers.

Some problems will make use of the \GRno{} suite discussed in class. These problems are available directly from Canvas
by clicking on each individual question.
You are allowed to submit code to \GRno{} as many times as you want before the due date without penalty. In this
way you can be sure each function or script passes all of the assesments that go along with it before moving on to the
next problem. We encourage you to write
your own test cases as well to ensure your code is working as expected. 

When writing functions and running simulations, use the set of parameters given to you for Niki. These will be available
in each \GRno{} problem where they are appropriate. Note that some problems will use different parameters than Niki's
true values to demonstrate a different vehicle dynamics concept. We'll try to point out where that happens. The real
values for Niki are given in Appendix \ref{appendix:Vehicle Parameters} and
\ref{appendix:Tire Parameters} at the end of this document.

This week we will additionally ask you to use a new tool \textbf{MATLAB Online}. This tool will be useful for simulating your project code without some of the limitations of \GRno{}. If you have a normal desktop license for MATLAB, you are welcome to use that as well, it will effectively be the same tool. See our Piazza post for more information about how to setup MATLAB Online.


\newpage
