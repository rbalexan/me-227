\secPr{Lateral Velocity as a Function of Speed}

The lateral velocity can be a little harder to understand intuitively than the yaw rate (which is just angular velocity).  In a steady corner, the lateral velocity describes the orientation of the vehicle centerline relative to the circular path the vehicle is taking.  This can be important to understand for automated vehicles since sensors are often attached to the vehicle and if the vehicle is not pointing straight along the path, we will need to compensate in some way.  In the last assignment, you noticed that the lateral velocity of the vehicle changes sign as the speed increases.  Let’s look at this in a little more detail.

\iftoggle{condensed}{}{
    \vspace*{0.5cm}
}

The transfer function for lateral velocity with respect to steer angle is given by:
\begin{align*}
\frac{U_y(s)}{\Delta(s)} = \frac{\frac{C_{\alpha f}I_z}{m}s+\frac{LbC_{\alpha r}C_{\alpha f}}{mU_x}-aC_{\alpha f}U_x}{
I_zs^2 + \left(\frac{I_z(C_{af}+C_{ar})}{mU_x} + \frac{a^2C_{af}+b^2C_{ar}}{U_x}\right)s + 
\left(\frac{C_{af}C_{ar}L^2}{mU_x^2} +bC_{ar}-aC_{af}\right)}
\end{align*}
The denominator of this transfer function is the same as the yaw rate transfer function we covered in class. This should make sense, given the relationship between steer angle and these state. In this problem you will use this function to relate lateral and longitudinal speed.

\iftoggle{condensed}{}{
    \newpage
}