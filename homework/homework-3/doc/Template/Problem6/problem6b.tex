%---------------------------------------------------------------------------------
% PROBLEM 3 - PART B
%---------------------------------------------------------------------------------
\secGS{Natural Frequency and Damping Ratio}

At speeds of 20 m/s, 40 m/s and 60 m/s, calculate the natural frequency and damping ratio for each car. If we think of an “ideal” vehicle response as being fast and well damped, does one car have a clear advantage or do we see a trade-off between these attributes?

\iftoggle{condensed}{}{
    \vspace*{0.5cm}
}

\expect{Calculate $\omega_n$ and $\zeta$ and enter them below. Also answer the question.}

\iftoggle{condensed}{
    \vspace*{0.5cm}
}{
    \subsubsection*{Solution:}
}

\iftoggle{solution}{
    %---------------------------------------------------------------------------------
% PROBLEM 3 - PART B
%---------------------------------------------------------------------------------
\secGS{Natural Frequency and Damping Ratio}

At speeds of 20 m/s, 40 m/s and 60 m/s, calculate the natural frequency and damping ratio for each car. If we think of an “ideal” vehicle response as being fast and well damped, does one car have a clear advantage or do we see a trade-off between these attributes?

\iftoggle{condensed}{}{
    \vspace*{0.5cm}
}

\expect{Calculate $\omega_n$ and $\zeta$ and enter them below. Also answer the question.}

\iftoggle{condensed}{
    \vspace*{0.5cm}
}{
    \subsubsection*{Solution:}
}

\iftoggle{solution}{
    %---------------------------------------------------------------------------------
% PROBLEM 3 - PART B
%---------------------------------------------------------------------------------
\secGS{Natural Frequency and Damping Ratio}

At speeds of 20 m/s, 40 m/s and 60 m/s, calculate the natural frequency and damping ratio for each car. If we think of an “ideal” vehicle response as being fast and well damped, does one car have a clear advantage or do we see a trade-off between these attributes?

\iftoggle{condensed}{}{
    \vspace*{0.5cm}
}

\expect{Calculate $\omega_n$ and $\zeta$ and enter them below. Also answer the question.}

\iftoggle{condensed}{
    \vspace*{0.5cm}
}{
    \subsubsection*{Solution:}
}

\iftoggle{solution}{
    %---------------------------------------------------------------------------------
% PROBLEM 3 - PART B
%---------------------------------------------------------------------------------
\secGS{Natural Frequency and Damping Ratio}

At speeds of 20 m/s, 40 m/s and 60 m/s, calculate the natural frequency and damping ratio for each car. If we think of an “ideal” vehicle response as being fast and well damped, does one car have a clear advantage or do we see a trade-off between these attributes?

\iftoggle{condensed}{}{
    \vspace*{0.5cm}
}

\expect{Calculate $\omega_n$ and $\zeta$ and enter them below. Also answer the question.}

\iftoggle{condensed}{
    \vspace*{0.5cm}
}{
    \subsubsection*{Solution:}
}

\iftoggle{solution}{
    \input{Solutions/Problem6/problem6b.tex}
    \newpage
}

\iftoggle{template}{
        \vspace*{13cm}
        \begin{solutionorbox}[1.5in]
        \end{solutionorbox}
        \newpage
}

\iftoggle{student}{
%---------------------------------------------------------------------------------
% STUDENT: BEGIN WORK
%---------------------------------------------------------------------------------


% Please box final answer
\textbf{Car B:}
\begin{center}
    \begin{tabular}{|c|c|c|c|}
        \hline
         & 20 m/s & 40 m/s & 60 m/s \\
        \hline
        Natural frequency $\omega_n$ &  rad/s &  rad/s &  rad/s \\
        Damping ratio $\zeta$ &  &  &  \\
        \hline
    \end{tabular}
\end{center}

\textbf{Car C:}
\begin{center}
    \begin{tabular}{|c|c|c|c|}
        \hline
         & 20 m/s & 40 m/s & 60 m/s \\
        \hline
        Natural frequency $\omega_n$ &  rad/s &  rad/s &  rad/s \\
        Damping ratio $\zeta$ &  &  &  \\
        \hline
    \end{tabular}
\end{center}

Answer: 
%---------------------------------------------------------------------------------
% STUDENT: END WORK
%---------------------------------------------------------------------------------
    \newpage
}


    \newpage
}

\iftoggle{template}{
        \vspace*{13cm}
        \begin{solutionorbox}[1.5in]
        \end{solutionorbox}
        \newpage
}

\iftoggle{student}{
%---------------------------------------------------------------------------------
% STUDENT: BEGIN WORK
%---------------------------------------------------------------------------------


% Please box final answer
\textbf{Car B:}
\begin{center}
    \begin{tabular}{|c|c|c|c|}
        \hline
         & 20 m/s & 40 m/s & 60 m/s \\
        \hline
        Natural frequency $\omega_n$ &  rad/s &  rad/s &  rad/s \\
        Damping ratio $\zeta$ &  &  &  \\
        \hline
    \end{tabular}
\end{center}

\textbf{Car C:}
\begin{center}
    \begin{tabular}{|c|c|c|c|}
        \hline
         & 20 m/s & 40 m/s & 60 m/s \\
        \hline
        Natural frequency $\omega_n$ &  rad/s &  rad/s &  rad/s \\
        Damping ratio $\zeta$ &  &  &  \\
        \hline
    \end{tabular}
\end{center}

Answer: 
%---------------------------------------------------------------------------------
% STUDENT: END WORK
%---------------------------------------------------------------------------------
    \newpage
}


    \newpage
}

\iftoggle{template}{
        \vspace*{13cm}
        \begin{solutionorbox}[1.5in]
        \end{solutionorbox}
        \newpage
}

\iftoggle{student}{
%---------------------------------------------------------------------------------
% STUDENT: BEGIN WORK
%---------------------------------------------------------------------------------


% Please box final answer
\textbf{Car B:}
\begin{center}
    \begin{tabular}{|c|c|c|c|}
        \hline
         & 20 m/s & 40 m/s & 60 m/s \\
        \hline
        Natural frequency $\omega_n$ &  rad/s &  rad/s &  rad/s \\
        Damping ratio $\zeta$ &  &  &  \\
        \hline
    \end{tabular}
\end{center}

\textbf{Car C:}
\begin{center}
    \begin{tabular}{|c|c|c|c|}
        \hline
         & 20 m/s & 40 m/s & 60 m/s \\
        \hline
        Natural frequency $\omega_n$ &  rad/s &  rad/s &  rad/s \\
        Damping ratio $\zeta$ &  &  &  \\
        \hline
    \end{tabular}
\end{center}

Answer: 
%---------------------------------------------------------------------------------
% STUDENT: END WORK
%---------------------------------------------------------------------------------
    \newpage
}


    \newpage
}

\iftoggle{template}{
        \vspace*{13cm}
        \begin{solutionorbox}[1.5in]
        \end{solutionorbox}
        \newpage
}

\iftoggle{student}{
%---------------------------------------------------------------------------------
% STUDENT: BEGIN WORK
%---------------------------------------------------------------------------------


% Please box final answer
\textbf{Car B:}
\begin{center}
    \begin{tabular}{|c|c|c|c|}
        \hline
         & 20 m/s & 40 m/s & 60 m/s \\
        \hline
        Natural frequency $\omega_n$ &  rad/s &  rad/s &  rad/s \\
        Damping ratio $\zeta$ &  &  &  \\
        \hline
    \end{tabular}
\end{center}

\textbf{Car C:}
\begin{center}
    \begin{tabular}{|c|c|c|c|}
        \hline
         & 20 m/s & 40 m/s & 60 m/s \\
        \hline
        Natural frequency $\omega_n$ &  rad/s &  rad/s &  rad/s \\
        Damping ratio $\zeta$ &  &  &  \\
        \hline
    \end{tabular}
\end{center}

Answer: 
%---------------------------------------------------------------------------------
% STUDENT: END WORK
%---------------------------------------------------------------------------------
    \newpage
}

