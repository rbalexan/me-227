\secPr{Poles and Speed Variation}

\textbf{\color{red} PROBLEM WRITTEN FOR 2020 BUT NOT USED}

Understeering vehicles have the desirable property that their yaw response never goes unstable but they do become increasingly oscillatory at higher speeds. This may not be completely apparent at U.S. highway speeds unless the understeer is quite high but can become quite pronounced at speeds reached on the German Autobahn. To get a sense for this, we are going to look at pole locations and simulations for two vehicles, Car B in the previous problem and a vehicle with a very closely balanced weight distribution (Car C). Properties for each car are given in the below table. Tire stiffnesses are the same for the front and rear.

\begin{center}
    \begin{tabular}{|c|c|c|}
        \hline
         & Car B & Car C \\
        \hline
        Wheelbase & 2.4 m & 2.4 m \\
        Mass & 1600 kg & 1600 kg \\
        $I_z$ & 2200 kg$\cdot$ m$^2$ & 2200 kg$\cdot$ m$^2$ \\
        Weight Distribution & 65/35 & 52/48\\
        Tire Cornering Stiffness & 150,000 N/rad & 150,000 N/rad \\
        \hline
    \end{tabular}
\end{center}

\iftoggle{condensed}{}{
    \vspace*{2cm}
}
