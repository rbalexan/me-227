\secGS{Calculating Dynamic Properties}

For each vehicle, calculate the characteristic speed and then the steady-state yaw gain, natural frequency and damping ratio at a speed of 65mph (be sure to convert to m/s first). The properties are given again below:

\begin{center}
    \begin{tabular}{|c|c|c|}
        \hline
         & Car A & Car B \\
        \hline
        Wheelbase & 2.4 m & 2.4 m \\
        Mass & 1600 kg & 1600 kg \\
        $I_z$ & 2200 kg$\cdot$ m$^2$ & 2200 kg$\cdot$ m$^2$ \\
        Weight Distribution & 60/40 & 65/35 \\
        Tire Cornering Stiffness & 100,000 N/rad & 150,000 N/rad \\
        \hline
    \end{tabular}
\end{center}

\textbf{Hints:
\begin{itemize}
    \item Make sure your understeer gradient, $K$, is in units of rad/m/s$^2$.
    \item You can use Matlab online (matlab.mathworks.com) to write reusable functions to reduce the risk of algebra mistakes.
\end{itemize}
}

\iftoggle{condensed}{}{
    \vspace*{0.5cm}
}

\expect{Calculate the characteristic speed, steady state yaw gain, natural frequency, and damping ratio. Put your final solutions below.}

\iftoggle{condensed}{
    \vspace*{0.5cm}
}{
    \subsubsection*{Solution:}
}

\iftoggle{solution}{
    %---------------------------------------------------------------------------------
% QUESTION 1.C
%--------------------------------------------------------------------------------
\secGS{Nonlinear Tire Phase Portrait}

Generate a phase portrait for $U_y$ and $r$ and include your plot.  How many equilibria are there? How would you classify these equilibria? Are they stable? Explain why.

\vspace*{0.5cm}

\expect{Include your phase potrait. How many equilibria are there? Describe what types of equilibria these are and if they are stable. Explain your reasoning.}


\iftoggle{condensed}{
    \vspace*{0.5cm}
}{
    \subsubsection*{Solution:}
}

\iftoggle{solution}{
    %---------------------------------------------------------------------------------
% QUESTION 1.C
%--------------------------------------------------------------------------------
\secGS{Nonlinear Tire Phase Portrait}

Generate a phase portrait for $U_y$ and $r$ and include your plot.  How many equilibria are there? How would you classify these equilibria? Are they stable? Explain why.

\vspace*{0.5cm}

\expect{Include your phase potrait. How many equilibria are there? Describe what types of equilibria these are and if they are stable. Explain your reasoning.}


\iftoggle{condensed}{
    \vspace*{0.5cm}
}{
    \subsubsection*{Solution:}
}

\iftoggle{solution}{
    %---------------------------------------------------------------------------------
% QUESTION 1.C
%--------------------------------------------------------------------------------
\secGS{Nonlinear Tire Phase Portrait}

Generate a phase portrait for $U_y$ and $r$ and include your plot.  How many equilibria are there? How would you classify these equilibria? Are they stable? Explain why.

\vspace*{0.5cm}

\expect{Include your phase potrait. How many equilibria are there? Describe what types of equilibria these are and if they are stable. Explain your reasoning.}


\iftoggle{condensed}{
    \vspace*{0.5cm}
}{
    \subsubsection*{Solution:}
}

\iftoggle{solution}{
    \input{Solutions/Problem1/problem1c.tex}
}

\iftoggle{template}{
    \begin{solutionorbox}[3.5in]
    \end{solutionorbox}
}

\iftoggle{student}{
%---------------------------------------------------------------------------------
% STUDENT: BEGIN WORK
%---------------------------------------------------------------------------------
% Please box final answer

%---------------------------------------------------------------------------------
% STUDENT: END WORK
%---------------------------------------------------------------------------------
}

\newpage


}

\iftoggle{template}{
    \begin{solutionorbox}[3.5in]
    \end{solutionorbox}
}

\iftoggle{student}{
%---------------------------------------------------------------------------------
% STUDENT: BEGIN WORK
%---------------------------------------------------------------------------------
% Please box final answer

%---------------------------------------------------------------------------------
% STUDENT: END WORK
%---------------------------------------------------------------------------------
}

\newpage


}

\iftoggle{template}{
    \begin{solutionorbox}[3.5in]
    \end{solutionorbox}
}

\iftoggle{student}{
%---------------------------------------------------------------------------------
% STUDENT: BEGIN WORK
%---------------------------------------------------------------------------------
% Please box final answer

%---------------------------------------------------------------------------------
% STUDENT: END WORK
%---------------------------------------------------------------------------------
}

\newpage


    \newpage
}

\iftoggle{template}{
        \vspace*{10cm}
        \begin{solutionorbox}[1.5in]
        \end{solutionorbox}
        \newpage
}

\iftoggle{student}{
%---------------------------------------------------------------------------------
% STUDENT: BEGIN WORK
%---------------------------------------------------------------------------------

% Please box final answer
\begin{center}
    \setlength{\tabcolsep}{36pt}
    \begin{tabular}{|c|c|c|}
        \hline
         & Car A & Car B \\
        \hline
        Characteristic speed &  &  \\
        Steady-state yaw gain &  &   \\
        $\omega_n$ &   &   \\
        $\zeta$ &  &  \\
        % Characteristic speed & m/s &  m/s \\
        % Steady-state yaw gain &  s$^{-1}$ &  s$^{-1}$ \\
        % $\omega_n$ &  rad/s &  rad/s \\
        % $\zeta$ &  &  \\
        \hline
    \end{tabular}
\end{center}

%---------------------------------------------------------------------------------
% STUDENT: END WORK
%---------------------------------------------------------------------------------
    \newpage
}

