\secGS{Calculating the Understeer Gradient}

Consider the following vehicles:

\begin{center}
    \begin{tabular}{|c|c|c|}
        \hline
         & Car A & Car B \\
        \hline
        Wheelbase & 2.4 m & 2.4 m \\
        Mass & 1600 kg & 1600 kg \\
        $I_z$ & 2200 kg$\cdot$ m$^2$ & 2200 kg$\cdot$ m$^2$ \\
        Weight Distribution & 60/40 & 65/35 \\
        Tire Cornering Stiffness & 100,000 N/rad & 150,000 N/rad \\
        \hline
    \end{tabular}
\end{center}

Verify that both cars have the same understeer gradient. Converting to degrees/g, where does the value of understeer gradient fall in a range of 0-3 deg/g with 0 representing a neutral steering car and 3 representing a fairly high degree of understeer?

\iftoggle{condensed}{}{
    \vspace*{0.5cm}
}

\expect{Show your work verifying that the two cars have the same understeer gradient. Write that understeer gradient below and answer where the understeer gradient falls in that range.}

\iftoggle{condensed}{
    \vspace*{0.5cm}
}{
    \subsubsection*{Solution:}
}

\iftoggle{solution}{
    \secGS{Calculating the Understeer Gradient}

Consider the following vehicles:

\begin{center}
    \begin{tabular}{|c|c|c|}
        \hline
         & Car A & Car B \\
        \hline
        Wheelbase & 2.4 m & 2.4 m \\
        Mass & 1600 kg & 1600 kg \\
        $I_z$ & 2200 kg$\cdot$ m$^2$ & 2200 kg$\cdot$ m$^2$ \\
        Weight Distribution & 60/40 & 65/35 \\
        Tire Cornering Stiffness & 100,000 N/rad & 150,000 N/rad \\
        \hline
    \end{tabular}
\end{center}

Verify that both cars have the same understeer gradient. Converting to degrees/g, where does the value of understeer gradient fall in a range of 0-3 deg/g with 0 representing a neutral steering car and 3 representing a fairly high degree of understeer?

\iftoggle{condensed}{}{
    \vspace*{0.5cm}
}

\expect{Show your work verifying that the two cars have the same understeer gradient. Write that understeer gradient below and answer where the understeer gradient falls in that range.}

\iftoggle{condensed}{
    \vspace*{0.5cm}
}{
    \subsubsection*{Solution:}
}

\iftoggle{solution}{
    \secGS{Calculating the Understeer Gradient}

Consider the following vehicles:

\begin{center}
    \begin{tabular}{|c|c|c|}
        \hline
         & Car A & Car B \\
        \hline
        Wheelbase & 2.4 m & 2.4 m \\
        Mass & 1600 kg & 1600 kg \\
        $I_z$ & 2200 kg$\cdot$ m$^2$ & 2200 kg$\cdot$ m$^2$ \\
        Weight Distribution & 60/40 & 65/35 \\
        Tire Cornering Stiffness & 100,000 N/rad & 150,000 N/rad \\
        \hline
    \end{tabular}
\end{center}

Verify that both cars have the same understeer gradient. Converting to degrees/g, where does the value of understeer gradient fall in a range of 0-3 deg/g with 0 representing a neutral steering car and 3 representing a fairly high degree of understeer?

\iftoggle{condensed}{}{
    \vspace*{0.5cm}
}

\expect{Show your work verifying that the two cars have the same understeer gradient. Write that understeer gradient below and answer where the understeer gradient falls in that range.}

\iftoggle{condensed}{
    \vspace*{0.5cm}
}{
    \subsubsection*{Solution:}
}

\iftoggle{solution}{
    \secGS{Calculating the Understeer Gradient}

Consider the following vehicles:

\begin{center}
    \begin{tabular}{|c|c|c|}
        \hline
         & Car A & Car B \\
        \hline
        Wheelbase & 2.4 m & 2.4 m \\
        Mass & 1600 kg & 1600 kg \\
        $I_z$ & 2200 kg$\cdot$ m$^2$ & 2200 kg$\cdot$ m$^2$ \\
        Weight Distribution & 60/40 & 65/35 \\
        Tire Cornering Stiffness & 100,000 N/rad & 150,000 N/rad \\
        \hline
    \end{tabular}
\end{center}

Verify that both cars have the same understeer gradient. Converting to degrees/g, where does the value of understeer gradient fall in a range of 0-3 deg/g with 0 representing a neutral steering car and 3 representing a fairly high degree of understeer?

\iftoggle{condensed}{}{
    \vspace*{0.5cm}
}

\expect{Show your work verifying that the two cars have the same understeer gradient. Write that understeer gradient below and answer where the understeer gradient falls in that range.}

\iftoggle{condensed}{
    \vspace*{0.5cm}
}{
    \subsubsection*{Solution:}
}

\iftoggle{solution}{
    \input{Solutions/Problem1/problem1b.tex}
    \newpage
}

\iftoggle{template}{
        \vspace*{9cm}
        \begin{solutionorbox}[1.5in]
        \end{solutionorbox}
        \newpage
}

\iftoggle{student}{
%---------------------------------------------------------------------------------
% STUDENT: BEGIN WORK
%---------------------------------------------------------------------------------

% Please box final answer

%---------------------------------------------------------------------------------
% STUDENT: END WORK
%---------------------------------------------------------------------------------
    \newpage
}


    \newpage
}

\iftoggle{template}{
        \vspace*{9cm}
        \begin{solutionorbox}[1.5in]
        \end{solutionorbox}
        \newpage
}

\iftoggle{student}{
%---------------------------------------------------------------------------------
% STUDENT: BEGIN WORK
%---------------------------------------------------------------------------------

% Please box final answer

%---------------------------------------------------------------------------------
% STUDENT: END WORK
%---------------------------------------------------------------------------------
    \newpage
}


    \newpage
}

\iftoggle{template}{
        \vspace*{9cm}
        \begin{solutionorbox}[1.5in]
        \end{solutionorbox}
        \newpage
}

\iftoggle{student}{
%---------------------------------------------------------------------------------
% STUDENT: BEGIN WORK
%---------------------------------------------------------------------------------

% Please box final answer

%---------------------------------------------------------------------------------
% STUDENT: END WORK
%---------------------------------------------------------------------------------
    \newpage
}


    \newpage
}

\iftoggle{template}{
        \vspace*{9cm}
        \begin{solutionorbox}[1.5in]
        \end{solutionorbox}
        \newpage
}

\iftoggle{student}{
%---------------------------------------------------------------------------------
% STUDENT: BEGIN WORK
%---------------------------------------------------------------------------------

% Please box final answer

%---------------------------------------------------------------------------------
% STUDENT: END WORK
%---------------------------------------------------------------------------------
    \newpage
}

