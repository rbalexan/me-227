\secGS{Calculating the Understeer Gradient}

Consider the following vehicles:

\begin{center}
    \begin{tabular}{|c|c|c|}
        \hline
         & Car A & Car B \\
        \hline
        Wheelbase & 2.4 m & 2.4 m \\
        Mass & 1600 kg & 1600 kg \\
        $I_z$ & 2200 kg$\cdot$ m$^2$ & 2200 kg$\cdot$ m$^2$ \\
        Weight Distribution & 60/40 & 65/35 \\
        Tire Cornering Stiffness & 100,000 N/rad & 150,000 N/rad \\
        \hline
    \end{tabular}
\end{center}

Verify that both cars have the same understeer gradient. Converting to degrees/g, where does the value of understeer gradient fall in a range of 0-3 deg/g with 0 representing a neutral steering car and 3 representing a fairly high degree of understeer?

\iftoggle{condensed}{}{
    \vspace*{0.5cm}
}

\expect{Show your work verifying that the two cars have the same understeer gradient. Write that understeer gradient below and answer where the understeer gradient falls in that range.}

\iftoggle{condensed}{
    \vspace*{0.5cm}
}{
    \subsubsection*{Solution:}
}

\iftoggle{solution}{
    %---------------------------------------------------------------------------------
% QUESTION 1.B
%---------------------------------------------------------------------------------

\secGR{Incorporating Path Tracking}

Drifting by itself is pretty cool, but what if we wanted to incorporate a reference path? Let's define a new state $x = [r, U_y, e, \Delta\psi]^T$.

%Since we have seen a lot of success with the lookahead controller, let's try adding an additional term to our steering command from Problem 2E:
%$$\delta=k_r(r_{\text{des}}-r)+k_y(U_{y,\text{des}}-U_y)+\delta_{ff}+\delta_{lookahead}$$
%where $\delta_{lookahead}$ is the lookahead controller we have been using all quarter (do not include curvature compensation).

\textbf{Find a value for $\Delta \psi_{eq}$ given the drift equilibrium we are using in this problem. Do not make any small angle assumptions.}

Let's linearize these dynamics as before and then try to find gains that stabilize the system. This time, our linearized matrices are:

\begin{equation}
A = 
\begin{bmatrix}
\frac{\partial \dot r}{\partial r}  & \frac{\partial \dot r}{\partial U_y}  & \frac{\partial \dot r}{\partial e} & \frac{\partial \dot r}{\partial \Delta\psi} \\
\frac{\partial \dot U_y}{\partial r}  & \frac{\partial \dot U_y}{\partial U_y}  & \frac{\partial \dot U_y}{\partial e} & \frac{\partial \dot U_y}{\partial \Delta\psi} \\
\frac{\partial \dot e}{\partial r}  & \frac{\partial \dot e}{\partial U_y}  & \frac{\partial \dot e}{\partial e} & \frac{\partial \dot e}{\partial \Delta\psi} \\
\frac{\partial \dot{\Delta\psi}}{\partial r}  & \frac{\partial \dot{\Delta\psi}}{\partial U_y}  & \frac{\partial \dot{\Delta\psi}}{\partial e} & \frac{\partial \dot{\Delta\psi}}{\partial \Delta\psi} \\
\end{bmatrix}_{x=x_{eq}} 
\end{equation}

\begin{equation}
B = \begin{bmatrix}
\frac{\partial \dot r}{\partial \delta} \\
\frac{\partial \dot U_y}{\partial \delta}\\
\frac{\partial \dot e}{\partial \delta} \\
\frac{\partial \dot \Delta\psi}{\partial \delta}\\
\end{bmatrix}_{x=x_{eq}} 
\end{equation}


\textbf{Find $\Delta{\psi_{ss}}$ and the new linearized dynamics matrices. Using full state feedback as before, place the poles at [-4 +/- 4.5j, -0.15 +/- 0.75j]. You are encouraged to use the 'place' function in MATLAB. Report your gain vector and $\Delta{\psi_{ss}}$ on MATLAB Grader}





\iftoggle{condensed}{
    \vspace*{0.5cm}
}{
    \subsubsection*{Solution:}
}

\iftoggle{solution}{
    %---------------------------------------------------------------------------------
% QUESTION 1.B
%---------------------------------------------------------------------------------

\secGR{Incorporating Path Tracking}

Drifting by itself is pretty cool, but what if we wanted to incorporate a reference path? Let's define a new state $x = [r, U_y, e, \Delta\psi]^T$.

%Since we have seen a lot of success with the lookahead controller, let's try adding an additional term to our steering command from Problem 2E:
%$$\delta=k_r(r_{\text{des}}-r)+k_y(U_{y,\text{des}}-U_y)+\delta_{ff}+\delta_{lookahead}$$
%where $\delta_{lookahead}$ is the lookahead controller we have been using all quarter (do not include curvature compensation).

\textbf{Find a value for $\Delta \psi_{eq}$ given the drift equilibrium we are using in this problem. Do not make any small angle assumptions.}

Let's linearize these dynamics as before and then try to find gains that stabilize the system. This time, our linearized matrices are:

\begin{equation}
A = 
\begin{bmatrix}
\frac{\partial \dot r}{\partial r}  & \frac{\partial \dot r}{\partial U_y}  & \frac{\partial \dot r}{\partial e} & \frac{\partial \dot r}{\partial \Delta\psi} \\
\frac{\partial \dot U_y}{\partial r}  & \frac{\partial \dot U_y}{\partial U_y}  & \frac{\partial \dot U_y}{\partial e} & \frac{\partial \dot U_y}{\partial \Delta\psi} \\
\frac{\partial \dot e}{\partial r}  & \frac{\partial \dot e}{\partial U_y}  & \frac{\partial \dot e}{\partial e} & \frac{\partial \dot e}{\partial \Delta\psi} \\
\frac{\partial \dot{\Delta\psi}}{\partial r}  & \frac{\partial \dot{\Delta\psi}}{\partial U_y}  & \frac{\partial \dot{\Delta\psi}}{\partial e} & \frac{\partial \dot{\Delta\psi}}{\partial \Delta\psi} \\
\end{bmatrix}_{x=x_{eq}} 
\end{equation}

\begin{equation}
B = \begin{bmatrix}
\frac{\partial \dot r}{\partial \delta} \\
\frac{\partial \dot U_y}{\partial \delta}\\
\frac{\partial \dot e}{\partial \delta} \\
\frac{\partial \dot \Delta\psi}{\partial \delta}\\
\end{bmatrix}_{x=x_{eq}} 
\end{equation}


\textbf{Find $\Delta{\psi_{ss}}$ and the new linearized dynamics matrices. Using full state feedback as before, place the poles at [-4 +/- 4.5j, -0.15 +/- 0.75j]. You are encouraged to use the 'place' function in MATLAB. Report your gain vector and $\Delta{\psi_{ss}}$ on MATLAB Grader}





\iftoggle{condensed}{
    \vspace*{0.5cm}
}{
    \subsubsection*{Solution:}
}

\iftoggle{solution}{
    %---------------------------------------------------------------------------------
% QUESTION 1.B
%---------------------------------------------------------------------------------

\secGR{Incorporating Path Tracking}

Drifting by itself is pretty cool, but what if we wanted to incorporate a reference path? Let's define a new state $x = [r, U_y, e, \Delta\psi]^T$.

%Since we have seen a lot of success with the lookahead controller, let's try adding an additional term to our steering command from Problem 2E:
%$$\delta=k_r(r_{\text{des}}-r)+k_y(U_{y,\text{des}}-U_y)+\delta_{ff}+\delta_{lookahead}$$
%where $\delta_{lookahead}$ is the lookahead controller we have been using all quarter (do not include curvature compensation).

\textbf{Find a value for $\Delta \psi_{eq}$ given the drift equilibrium we are using in this problem. Do not make any small angle assumptions.}

Let's linearize these dynamics as before and then try to find gains that stabilize the system. This time, our linearized matrices are:

\begin{equation}
A = 
\begin{bmatrix}
\frac{\partial \dot r}{\partial r}  & \frac{\partial \dot r}{\partial U_y}  & \frac{\partial \dot r}{\partial e} & \frac{\partial \dot r}{\partial \Delta\psi} \\
\frac{\partial \dot U_y}{\partial r}  & \frac{\partial \dot U_y}{\partial U_y}  & \frac{\partial \dot U_y}{\partial e} & \frac{\partial \dot U_y}{\partial \Delta\psi} \\
\frac{\partial \dot e}{\partial r}  & \frac{\partial \dot e}{\partial U_y}  & \frac{\partial \dot e}{\partial e} & \frac{\partial \dot e}{\partial \Delta\psi} \\
\frac{\partial \dot{\Delta\psi}}{\partial r}  & \frac{\partial \dot{\Delta\psi}}{\partial U_y}  & \frac{\partial \dot{\Delta\psi}}{\partial e} & \frac{\partial \dot{\Delta\psi}}{\partial \Delta\psi} \\
\end{bmatrix}_{x=x_{eq}} 
\end{equation}

\begin{equation}
B = \begin{bmatrix}
\frac{\partial \dot r}{\partial \delta} \\
\frac{\partial \dot U_y}{\partial \delta}\\
\frac{\partial \dot e}{\partial \delta} \\
\frac{\partial \dot \Delta\psi}{\partial \delta}\\
\end{bmatrix}_{x=x_{eq}} 
\end{equation}


\textbf{Find $\Delta{\psi_{ss}}$ and the new linearized dynamics matrices. Using full state feedback as before, place the poles at [-4 +/- 4.5j, -0.15 +/- 0.75j]. You are encouraged to use the 'place' function in MATLAB. Report your gain vector and $\Delta{\psi_{ss}}$ on MATLAB Grader}





\iftoggle{condensed}{
    \vspace*{0.5cm}
}{
    \subsubsection*{Solution:}
}

\iftoggle{solution}{
    \input{Solutions/Problem1/problem1b.tex}
}

\iftoggle{template}{
    \begin{solutionorbox}[5in]
    \end{solutionorbox}
    
    \newpage
}

\iftoggle{student}{
%---------------------------------------------------------------------------------
% STUDENT: BEGIN WORK
%---------------------------------------------------------------------------------
% Please box final answer

%---------------------------------------------------------------------------------
% STUDENT: END WORK
%---------------------------------------------------------------------------------
\newpage
}


}

\iftoggle{template}{
    \begin{solutionorbox}[5in]
    \end{solutionorbox}
    
    \newpage
}

\iftoggle{student}{
%---------------------------------------------------------------------------------
% STUDENT: BEGIN WORK
%---------------------------------------------------------------------------------
% Please box final answer

%---------------------------------------------------------------------------------
% STUDENT: END WORK
%---------------------------------------------------------------------------------
\newpage
}


}

\iftoggle{template}{
    \begin{solutionorbox}[5in]
    \end{solutionorbox}
    
    \newpage
}

\iftoggle{student}{
%---------------------------------------------------------------------------------
% STUDENT: BEGIN WORK
%---------------------------------------------------------------------------------
% Please box final answer

%---------------------------------------------------------------------------------
% STUDENT: END WORK
%---------------------------------------------------------------------------------
\newpage
}


    \newpage
}

\iftoggle{template}{
        \vspace*{9cm}
        \begin{solutionorbox}[1.5in]
        \end{solutionorbox}
        \newpage
}

\iftoggle{student}{
%---------------------------------------------------------------------------------
% STUDENT: BEGIN WORK
%---------------------------------------------------------------------------------

% Please box final answer

%---------------------------------------------------------------------------------
% STUDENT: END WORK
%---------------------------------------------------------------------------------
    \newpage
}

