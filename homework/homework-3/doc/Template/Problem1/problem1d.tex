\secGS{Comparison to the Kinematic Model}

For each vehicle, you calculated the steady-state yaw gain. Calculate the steady-state yaw gain predicted by the kinematic model at 65mph (hint: it is the same for both vehicles). You can use a small angle assumption. The properties are given again below:

\begin{center}
    \begin{tabular}{|c|c|c|}
        \hline
         & Car A & Car B \\
        \hline
        Wheelbase & 2.4 m & 2.4 m \\
        Mass & 1600 kg & 1600 kg \\
        $I_z$ & 2200 kg$\cdot$ m$^2$ & 2200 kg$\cdot$ m$^2$ \\
        Weight Distribution & 60/40 & 65/35 \\
        Tire Cornering Stiffness & 100,000 N/rad & 150,000 N/rad \\
        \hline
    \end{tabular}
\end{center}

How do the steady-state yaw gains predicted by each model compare?

\iftoggle{condensed}{}{
    \vspace*{0.5cm}
}

\expect{Calculate the kinematic model gain. Explain how it is different from the gains you calculated. Put your answers below.}

\iftoggle{condensed}{
    \vspace*{0.5cm}
}{
    \subsubsection*{Solution:}
}

\iftoggle{solution}{
    %---------------------------------------------------------------------------------
% QUESTION 1.D
%--------------------------------------------------------------------------------
\secGS{Lookahead Gain Pole Position Analysis}

Examine the plots you created in \qrefGR{Varying Lookahead Gain}. 
What are the trends in the natural frequency, $\omega_n$, and damping ratio, $\zeta$, of the \textbf{dominant pair of
poles} as
the lookahead gain is increased from 1,000 \si{\N/\m} to 10,000 \si{\N/\m}?

\vspace*{0.5cm}


\expect{Describe the behavior of the dominant pair of poles in terms of $\omega_n$ and $\zeta$ as lookahead gain is increased}.


\iftoggle{condensed}{
    \vspace*{0.5cm}
}{
    \subsubsection*{Solution:}
}

\iftoggle{solution}{
    %---------------------------------------------------------------------------------
% QUESTION 1.D
%--------------------------------------------------------------------------------
\secGS{Lookahead Gain Pole Position Analysis}

Examine the plots you created in \qrefGR{Varying Lookahead Gain}. 
What are the trends in the natural frequency, $\omega_n$, and damping ratio, $\zeta$, of the \textbf{dominant pair of
poles} as
the lookahead gain is increased from 1,000 \si{\N/\m} to 10,000 \si{\N/\m}?

\vspace*{0.5cm}


\expect{Describe the behavior of the dominant pair of poles in terms of $\omega_n$ and $\zeta$ as lookahead gain is increased}.


\iftoggle{condensed}{
    \vspace*{0.5cm}
}{
    \subsubsection*{Solution:}
}

\iftoggle{solution}{
    %---------------------------------------------------------------------------------
% QUESTION 1.D
%--------------------------------------------------------------------------------
\secGS{Lookahead Gain Pole Position Analysis}

Examine the plots you created in \qrefGR{Varying Lookahead Gain}. 
What are the trends in the natural frequency, $\omega_n$, and damping ratio, $\zeta$, of the \textbf{dominant pair of
poles} as
the lookahead gain is increased from 1,000 \si{\N/\m} to 10,000 \si{\N/\m}?

\vspace*{0.5cm}


\expect{Describe the behavior of the dominant pair of poles in terms of $\omega_n$ and $\zeta$ as lookahead gain is increased}.


\iftoggle{condensed}{
    \vspace*{0.5cm}
}{
    \subsubsection*{Solution:}
}

\iftoggle{solution}{
    \input{Solutions/Problem1/problem1d.tex}
    \newpage
}

\iftoggle{template}{
    \begin{solutionorbox}[3.5in]
    \end{solutionorbox}
    \newpage
}

\iftoggle{student}{
%---------------------------------------------------------------------------------
% STUDENT: BEGIN WORK
%---------------------------------------------------------------------------------
% Please box final answer

%---------------------------------------------------------------------------------
% STUDENT: END WORK
%---------------------------------------------------------------------------------
    \newpage
}


    \newpage
}

\iftoggle{template}{
    \begin{solutionorbox}[3.5in]
    \end{solutionorbox}
    \newpage
}

\iftoggle{student}{
%---------------------------------------------------------------------------------
% STUDENT: BEGIN WORK
%---------------------------------------------------------------------------------
% Please box final answer

%---------------------------------------------------------------------------------
% STUDENT: END WORK
%---------------------------------------------------------------------------------
    \newpage
}


    \newpage
}

\iftoggle{template}{
    \begin{solutionorbox}[3.5in]
    \end{solutionorbox}
    \newpage
}

\iftoggle{student}{
%---------------------------------------------------------------------------------
% STUDENT: BEGIN WORK
%---------------------------------------------------------------------------------
% Please box final answer

%---------------------------------------------------------------------------------
% STUDENT: END WORK
%---------------------------------------------------------------------------------
    \newpage
}


    \newpage
}

\iftoggle{template}{
        \vspace*{9cm}
        \begin{solutionorbox}[1.5in]
        \end{solutionorbox}
        \newpage
}

\iftoggle{student}{
%---------------------------------------------------------------------------------
% STUDENT: BEGIN WORK
%---------------------------------------------------------------------------------

% Please box final answer
Kinematic model yaw gain: 

Comparison:

%---------------------------------------------------------------------------------
% STUDENT: END WORK
%---------------------------------------------------------------------------------
    \newpage
}

