%---------------------------------------------------------------------------------
% PROBLEM 5 - PART A
%---------------------------------------------------------------------------------
\secGS{Closed Loop Transfer Function}

Eliminate the steering angle and show that the closed-loop transfer function of the driver and vehicle combined from desired yaw rate to yaw rate is:
\begin{align*}
\frac{R(s)}{R_{des}(s)} = \frac{K_{driver}\left(aC_{af}s+\frac{LC_{af}C_{ar}}{mU_x}\right)}{
I_zs^2 + \left(\frac{I_z(C_{af}+C_{ar})}{mU_x} + \frac{a^2C_{af}+b^2C_{ar}}{U_x} + aC_{af}K_{driver}\right)s + 
\left(\frac{C_{af}C_{ar}L^2}{mU_x^2} +bC_{ar}-aC_{af} + K_{driver}\frac{LC_{af}C_{ar}}{mU_x}\right)}
\end{align*}

\textit{Hint:}
You can make the algebra simpler on yourself by writing the transfer function as
\begin{align*}
    \frac{R(s)}{\Delta(s)} &= \frac{ls+m}{ns^2+ps+q}
\end{align*}
and doing the algebra before subbing the coefficients back in.

\iftoggle{condensed}{}{
    \vspace*{0.5cm}
}

\expect{Derive the closed loop transfer function given above.}

\iftoggle{condensed}{
    \vspace*{0.5cm}
}{
    \subsubsection*{Solution:}
}

\iftoggle{solution}{
    %---------------------------------------------------------------------------------
% QUESTION 3.A
%---------------------------------------------------------------------------------
\secMOGR{Implementing A Nonlinear Bicycle Model}

Using the code template provided, implement a function \verb!nonlinear_bicycle_model! that takes in vehicle states and
parameters and calculates the state derivatives for each of the states. You should implement all of the position and
velocity states in the model for a total of 6 states. Do not use small angle assumptions.

Niki is a front wheel drive vehicle. This means any positive longitudinal force (drive force) will be applied only at the front tires.
If the longitudinal force is negative (braking force), it will be split evenly between the front and rear axles. You need to implement
this behavior in your \verb!nonlinear_bicycle_model! function.

When you are finished writing this function in \MOno{}, copy and paste your code in the corresponding \GRno{}
prompt to validate your results. You can use the code in the "Code to Call Your Function" section of the \GRno{} prompt
to debug and develop your code in \MOno{}.

\iftoggle{solution}{
    \input{Solutions/matlabgrader.tex}
}

\newpage

    \newpage
}

\iftoggle{template}{
        \vspace*{8cm}
        \begin{solutionorbox}[1.5in]
        \end{solutionorbox}
        \newpage
}

\iftoggle{student}{
%---------------------------------------------------------------------------------
% STUDENT: BEGIN WORK
%---------------------------------------------------------------------------------


% Please box final answer

%---------------------------------------------------------------------------------
% STUDENT: END WORK
%---------------------------------------------------------------------------------
    \newpage
}

