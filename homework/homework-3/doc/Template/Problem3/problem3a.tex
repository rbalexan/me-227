%---------------------------------------------------------------------------------
% PROBLEM 5 - PART A
%---------------------------------------------------------------------------------
\secGS{Closed Loop Transfer Function}

Eliminate the steering angle and show that the closed-loop transfer function of the driver and vehicle combined from desired yaw rate to yaw rate is:
\begin{align*}
\frac{R(s)}{R_{des}(s)} = \frac{K_{driver}\left(aC_{af}s+\frac{LC_{af}C_{ar}}{mU_x}\right)}{
I_zs^2 + \left(\frac{I_z(C_{af}+C_{ar})}{mU_x} + \frac{a^2C_{af}+b^2C_{ar}}{U_x} + aC_{af}K_{driver}\right)s + 
\left(\frac{C_{af}C_{ar}L^2}{mU_x^2} +bC_{ar}-aC_{af} + K_{driver}\frac{LC_{af}C_{ar}}{mU_x}\right)}
\end{align*}

\textit{Hint:}
You can make the algebra simpler on yourself by writing the transfer function as
\begin{align*}
    \frac{R(s)}{\Delta(s)} &= \frac{ls+m}{ns^2+ps+q}
\end{align*}
and doing the algebra before subbing the coefficients back in.

\iftoggle{condensed}{}{
    \vspace*{0.5cm}
}

\expect{Derive the closed loop transfer function given above.}

\iftoggle{condensed}{
    \vspace*{0.5cm}
}{
    \subsubsection*{Solution:}
}

\iftoggle{solution}{
    %---------------------------------------------------------------------------------
% QUESTION 3.A
%---------------------------------------------------------------------------------
\secMOGS{Benchmarking Our Setup}

Begin with a brake distribution of 64\%/36\% between the front and rear axle, and roll stiffness Combination A (from Table
\ref{Table:RollParams}). If we want to be able to follow the path within 2m of the desired path, how are we doing at
this point? Which tire saturates at the start of the turn? Is this limit understeer or limit oversteer behavior?  What evidence do we have that this is the case?

\expect{Run the simulation and answer the questions about the vehicle's behavior. Include the tire force plots.}

\iftoggle{condensed}{
    \vspace*{0.5cm}
}{
    \subsubsection*{Solution:}
}

\iftoggle{solution}{
    %---------------------------------------------------------------------------------
% QUESTION 3.A
%---------------------------------------------------------------------------------
\secMOGS{Benchmarking Our Setup}

Begin with a brake distribution of 64\%/36\% between the front and rear axle, and roll stiffness Combination A (from Table
\ref{Table:RollParams}). If we want to be able to follow the path within 2m of the desired path, how are we doing at
this point? Which tire saturates at the start of the turn? Is this limit understeer or limit oversteer behavior?  What evidence do we have that this is the case?

\expect{Run the simulation and answer the questions about the vehicle's behavior. Include the tire force plots.}

\iftoggle{condensed}{
    \vspace*{0.5cm}
}{
    \subsubsection*{Solution:}
}

\iftoggle{solution}{
    %---------------------------------------------------------------------------------
% QUESTION 3.A
%---------------------------------------------------------------------------------
\secMOGS{Benchmarking Our Setup}

Begin with a brake distribution of 64\%/36\% between the front and rear axle, and roll stiffness Combination A (from Table
\ref{Table:RollParams}). If we want to be able to follow the path within 2m of the desired path, how are we doing at
this point? Which tire saturates at the start of the turn? Is this limit understeer or limit oversteer behavior?  What evidence do we have that this is the case?

\expect{Run the simulation and answer the questions about the vehicle's behavior. Include the tire force plots.}

\iftoggle{condensed}{
    \vspace*{0.5cm}
}{
    \subsubsection*{Solution:}
}

\iftoggle{solution}{
    \input{Solutions/Problem3/problem3a.tex}
    \newpage
}

\iftoggle{template}{
    \begin{solutionorbox}[3.5in]
    \end{solutionorbox}
    \newpage
}

\iftoggle{student}{
%---------------------------------------------------------------------------------
% STUDENT: BEGIN WORK
%---------------------------------------------------------------------------------


% Please box final answer

%---------------------------------------------------------------------------------
% STUDENT: END WORK
%---------------------------------------------------------------------------------
    \newpage
}


    \newpage
}

\iftoggle{template}{
    \begin{solutionorbox}[3.5in]
    \end{solutionorbox}
    \newpage
}

\iftoggle{student}{
%---------------------------------------------------------------------------------
% STUDENT: BEGIN WORK
%---------------------------------------------------------------------------------


% Please box final answer

%---------------------------------------------------------------------------------
% STUDENT: END WORK
%---------------------------------------------------------------------------------
    \newpage
}


    \newpage
}

\iftoggle{template}{
    \begin{solutionorbox}[3.5in]
    \end{solutionorbox}
    \newpage
}

\iftoggle{student}{
%---------------------------------------------------------------------------------
% STUDENT: BEGIN WORK
%---------------------------------------------------------------------------------


% Please box final answer

%---------------------------------------------------------------------------------
% STUDENT: END WORK
%---------------------------------------------------------------------------------
    \newpage
}


    \newpage
}

\iftoggle{template}{
        \vspace*{8cm}
        \begin{solutionorbox}[1.5in]
        \end{solutionorbox}
        \newpage
}

\iftoggle{student}{
%---------------------------------------------------------------------------------
% STUDENT: BEGIN WORK
%---------------------------------------------------------------------------------


% Please box final answer

%---------------------------------------------------------------------------------
% STUDENT: END WORK
%---------------------------------------------------------------------------------
    \newpage
}

