\secPr{Simulating and Tuning Vehicle Setup}

Through previous homeworks, you've built up a nonlinear vehicle simulator to analyze vehicle behavior and design
autonomous controllers. As you saw in the project, this simulator captures a good amount of detail of the vehicle's real
behavior, but not all of it. In this question we will add one more level of complexity by accounting for weight transfer
and roll effects. We have supplied all of the code you need for this Problem in the \verb!Problem3/! directory included
in the \verb!hw7.zip! file on Canvas. 

In this problem we will simulate Niki racing through the hairpin corner you saw in
\pref{How to Race Through the Hairpin Corner}. 
It is important to note that Niki is front wheel drive; this will frame your analysis of Niki's behavior through the
manuever. The simulation will run for 14 seconds. By keeping the simulation time the same, we can compare the value of
$s$ reached when the simulation ends. A higher value of $s$ means we made greater progress through the corner, and
consequently did a better job navigating the corner.


The simulator included here is essentially your nonlinear vehicle simulator with the weight transfer and effective
friction calculation functions you wrote in \pref{Calculating Weight Transfer and Effective Friction Coefficient}.  This is all conducted using the MF tire model, not the modified Fiala Model we've used in previous assignments. 

When asked you will just need to comment/uncomment different roll stiffness and brake distribution combinations in the
\verb!simulator.m! script and run the simulation to see how a different setup affects you performance through the
corner.  You will then need to implement aerodynamic effects in \verb!aero_effects.m! from a splitter and wing that we'll mount on Niki.

\iftoggle{condensed}{}{
    \newpage
}

