%---------------------------------------------------------------------------------
% QUESTION 2.C
%---------------------------------------------------------------------------------
\secGS{How Does Weight Transfer Affect Friction}

In the "Run Function" section of the \GRno{} assignment \qrefGR{Calculating Effective Friction}, there are three
combinations of roll stiffness as listed in Table \ref{Table:RollParams} below.

\renewcommand{\arraystretch}{1.25}
\begin{table}[h!]
    \centering
    \begin{tabular}{| c | c | c|}
        \hline
        \bld{Combination} & \bld{Parameter} & \bld{Value} \\[5pt]
        \hline\hline
        Combination A  &   Front Roll Stiffness $K_{\phi f}$   &    40,000 \si{\N\m/\radian} \\
                       &   Rear Roll Stiffness $K_{\phi r}$   &    80,000 \si{\N\m/\radian} \\\hline
        Combination B  &   Front Roll Stiffness $K_{\phi f}$   &    90,000 \si{\N\m/\radian} \\
                       &   Rear Roll Stiffness $K_{\phi r}$   &    30,000 \si{\N\m/\radian} \\\hline
        Combination C  &   Front Roll Stiffness $K_{\phi f}$   &    70,000 \si{\N\m/\radian} \\
                       &   Rear Roll Stiffness $K_{\phi r}$   &    50,000 \si{\N\m/\radian} \\\hline
    \end{tabular}
    \caption{Roll Stiffness Combinations}
    \label{Table:RollParams}
\end{table}
\renewcommand{\arraystretch}{1}

A quick way to gain intuition about the effects of roll stiffness is to see how the car behaves at a
constant lateral acceleration. Here we've set up a scenario where the vehicle is cornering at 0.9g. Run your function
and answer the following questions for each combination of roll stiffnesses:

\begin{enumerate}
    \item What is the roll angle at 0.9g lateral acceleration?
    \item How much load is on each tire when the vehicle is cornering at 0.9g?
    \item What are the front and rear friction coefficients when the vehicle is cornering at 0.9g?
\end{enumerate}


\vspace*{0.5cm}


\expect{Answer the questions about the vehicle's behavior for each combination of roll stiffnesses.}


\iftoggle{condensed}{
    \vspace*{0.5cm}
}{
    \subsubsection*{Solution:}
}

\iftoggle{solution}{
    %---------------------------------------------------------------------------------
% QUESTION 2.C
%---------------------------------------------------------------------------------
\secGS{Simulating from Equilibrium}

For this problem we will use the simple coupled tire model where $F_x$ is assumed to be known. Incorporate this into your simulation (We will supply verification code). Run your simulation of the three-state bicycle model for 4 seconds using $\delta$ = -10° and $F_{xr}$ equal to the value you computed in Problem 2B. Use the drift equilibrium found in Problem 1D as the initial condition. Plot $U_y$, $r$, and $U_x$ on the same plot. Does Marty hold the drift? What happens? Did you expect this? If you'd like, visualize using \texttt{animateDrift.m}.

\vspace*{0.5cm}


\expect{Include your plot and an explanation of what you see.}


\iftoggle{condensed}{
    \vspace*{0.5cm}
}{
    \subsubsection*{Solution:}
}

\iftoggle{solution}{
    %---------------------------------------------------------------------------------
% QUESTION 2.C
%---------------------------------------------------------------------------------
\secGS{Simulating from Equilibrium}

For this problem we will use the simple coupled tire model where $F_x$ is assumed to be known. Incorporate this into your simulation (We will supply verification code). Run your simulation of the three-state bicycle model for 4 seconds using $\delta$ = -10° and $F_{xr}$ equal to the value you computed in Problem 2B. Use the drift equilibrium found in Problem 1D as the initial condition. Plot $U_y$, $r$, and $U_x$ on the same plot. Does Marty hold the drift? What happens? Did you expect this? If you'd like, visualize using \texttt{animateDrift.m}.

\vspace*{0.5cm}


\expect{Include your plot and an explanation of what you see.}


\iftoggle{condensed}{
    \vspace*{0.5cm}
}{
    \subsubsection*{Solution:}
}

\iftoggle{solution}{
    %---------------------------------------------------------------------------------
% QUESTION 2.C
%---------------------------------------------------------------------------------
\secGS{Simulating from Equilibrium}

For this problem we will use the simple coupled tire model where $F_x$ is assumed to be known. Incorporate this into your simulation (We will supply verification code). Run your simulation of the three-state bicycle model for 4 seconds using $\delta$ = -10° and $F_{xr}$ equal to the value you computed in Problem 2B. Use the drift equilibrium found in Problem 1D as the initial condition. Plot $U_y$, $r$, and $U_x$ on the same plot. Does Marty hold the drift? What happens? Did you expect this? If you'd like, visualize using \texttt{animateDrift.m}.

\vspace*{0.5cm}


\expect{Include your plot and an explanation of what you see.}


\iftoggle{condensed}{
    \vspace*{0.5cm}
}{
    \subsubsection*{Solution:}
}

\iftoggle{solution}{
    \input{Solutions/Problem2/problem2c.tex}
}

\iftoggle{template}{
    \begin{solutionorbox}[6in]
    \end{solutionorbox}
    
    \newpage
}

\iftoggle{student}{
%---------------------------------------------------------------------------------
% STUDENT: BEGIN WORK
%---------------------------------------------------------------------------------
% Please box final answer

%---------------------------------------------------------------------------------
% STUDENT: END WORK
%---------------------------------------------------------------------------------
\newpage
}


}

\iftoggle{template}{
    \begin{solutionorbox}[6in]
    \end{solutionorbox}
    
    \newpage
}

\iftoggle{student}{
%---------------------------------------------------------------------------------
% STUDENT: BEGIN WORK
%---------------------------------------------------------------------------------
% Please box final answer

%---------------------------------------------------------------------------------
% STUDENT: END WORK
%---------------------------------------------------------------------------------
\newpage
}


}

\iftoggle{template}{
    \begin{solutionorbox}[6in]
    \end{solutionorbox}
    
    \newpage
}

\iftoggle{student}{
%---------------------------------------------------------------------------------
% STUDENT: BEGIN WORK
%---------------------------------------------------------------------------------
% Please box final answer

%---------------------------------------------------------------------------------
% STUDENT: END WORK
%---------------------------------------------------------------------------------
\newpage
}


}

\iftoggle{template}{
    \begin{solutionorbox}[3.5in]
    \end{solutionorbox}
}

\iftoggle{student}{
%---------------------------------------------------------------------------------
% STUDENT: BEGIN WORK
%---------------------------------------------------------------------------------


% Please box final answer

%---------------------------------------------------------------------------------
% STUDENT: END WORK
%---------------------------------------------------------------------------------
}

\newpage

