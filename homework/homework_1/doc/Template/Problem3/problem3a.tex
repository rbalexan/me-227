%---------------------------------------------------------------------------------
% PROBLEM 3 - PART A
%---------------------------------------------------------------------------------
\secGS{Lookahead Controller}
A common method for path tracking with vehicles is to use a lookahead controller. This can either look at the projected error some distance ahead on the path or, more commonly, approximate this error as a linear combination of the lateral error and heading error:

\al{
\delta &= K_{la} e_{la}\\
e_{la} &= e + d_{la} \Delta \psi
}

Here $d_{la}$ functions as the “lookahead distance” ahead of the reference point at which the error is calculated. This is exactly the projected error on a straight road and approximately the projected error on a curved road if the curvature is not too large relative to the lookahead distance.

During lecture, we also derived these equations assuming a small heading error and a small curvature:

\al{
\dot{s} &= V\\
\dot{e} &= V\Delta\psi\\
\dot{\Delta\psi} &= \frac{V\delta}L-\kappa V
}

Perform several Laplace transforms assuming small angles, a straight road, and \textbf{constant velocity}. Arrange this into a form where we can look at the response to the lateral error from the initial condition. In other words, calculate the relationship:

\eq{
\frac{E(s)}{e(0)}
}

The denominator of this relationship is the characteristic equation of the system that we can use to design the performance of our tracking controller.

Note: \textit{Where does the initial condition $e(0)$ come from? Remember that initial conditions are a part of the
    Laplace transformation, we just often set them to zero when forming a transfer function. Leave them in your expression
here.}

\vspace*{0.5cm}

\expect{Write the resulting relationship $\frac{E(s)}{e(0)}$. Please include your work.}


\iftoggle{condensed}{
    \vspace*{0.5cm}
}{
    \newpage
    \subsubsection*{Solution:}
}

\iftoggle{solution}{
    %---------------------------------------------------------------------------------
% QUESTION 3.A
%---------------------------------------------------------------------------------
\secMOGR{Implementing A Nonlinear Bicycle Model}

Using the code template provided, implement a function \verb!nonlinear_bicycle_model! that takes in vehicle states and
parameters and calculates the state derivatives for each of the states. You should implement all of the position and
velocity states in the model for a total of 6 states. Do not use small angle assumptions.

Niki is a front wheel drive vehicle. This means any positive longitudinal force (drive force) will be applied only at the front tires.
If the longitudinal force is negative (braking force), it will be split evenly between the front and rear axles. You need to implement
this behavior in your \verb!nonlinear_bicycle_model! function.

When you are finished writing this function in \MOno{}, copy and paste your code in the corresponding \GRno{}
prompt to validate your results. You can use the code in the "Code to Call Your Function" section of the \GRno{} prompt
to debug and develop your code in \MOno{}.

\iftoggle{solution}{
    \input{Solutions/matlabgrader.tex}
}

\newpage

}

\iftoggle{template}{
        \vspace*{17cm}
        \begin{solutionorbox}[1.5in]
        \end{solutionorbox}
}

\iftoggle{student}{
%---------------------------------------------------------------------------------
% STUDENT: BEGIN WORK
%---------------------------------------------------------------------------------


% Please box final answer

%---------------------------------------------------------------------------------
% STUDENT: END WORK
%---------------------------------------------------------------------------------
}

\newpage

